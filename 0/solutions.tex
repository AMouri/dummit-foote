\documentclass[12pt]{article}
\usepackage{amsmath, amssymb, listings}
\begin{document}
\title{Preliminaries}
\author{Alec Mouri}

\maketitle
\section{Basics}
\subsection*{Proposition 1}
\begin{itemize}
\item[(1)] Let $f$ be injective. Then for every $a \in A$, $f(a)$ is unique. Let $\beta = \left\lbrace f(a) \right\rbrace_{a \in A}$. Note that $\beta \subseteq B$. Define $g : B \rightarrow A$ as follows - for every $b \in \beta$ where $b = f(a)$ for some $a \in A$, then let $g(b) = a$. Moreover, for every $c \in B$ where $c \not \in \beta$, then $g(c)$ maps to some arbitrary element of $A$. $g \circ f$ is a mapping $A \rightarrow A$ where for $a \in A$, $(g \circ f)(a) = g(b) = a$. Therefore, $f$ has a left inverse $g$.

Suppose $f$ is not injective. Then for some two elements $a_1, a_2 \in A$, $f(a_1) = f(a_2) = b$. Clearly, $f$ does not have a left inverse - for every function $g : B \rightarrow A$, $g(b)$ can only map either to $a_1$ or $a_2$. Without loss of generality, consider $g(b) = a_1$. Then $(g \circ f)(a_2) = g(b) = a_1 \neq a_2$. Thus, $g \circ f$ is not the identity mapping on $A$.

\item[(2)] Let $f$ be surjective. Then for every $b \in B$, $\exists a \in A$ such that $f(a) = b$. That is, the fiber of $f$ over $b$ is nonempty. Define $h : B \rightarrow A$ as follows - for every $b \in B$, consider the fiber $\eta$ of $f$ over $b$. For some $a \in \eta$, then $h(b) = a$. $f \circ h$ is a mapping $B \rightarrow B$ where for $b \in B$, $(f \circ h)(b) = f(a) = b$. Therefore, $f$ has a right inverse $h$.

Suppose $f$ is not surjective. Then there exists some $b \in B$ where the fiber of $f$ over $b$ is empty. It is then clear that $f$ does not have a right inverse - $f \circ h$ cannot be the identity mapping on $B$ because although $b$ may be in the domain of $f \circ h$, $f$ may never output $b$.

\item[(3)] Let $f$ be a bijection. That is, $f$ is both injective and surjective. Then for every $a \in A$, $f(a)$ is unique. Moreover, for every $b \in B$, $\exists a \in A$ such that $f(a) = b$. Therefore, defne $g : B \rightarrow A$ as follows - for every $b \in B$ where $b = f(a)$ for some $a \in A$, then let $g(b) = a$. $g \circ f$ is a mapping $A \rightarrow A$ where for $a \in A$, $(g \circ f)(a) = g(b) = a$. So $g \circ f$ is the identity map on $A$. $f \circ g$ is a mapping $B \rightarrow B$ where for $b \in B$, $(f \circ g)(b) = f(a) = b$. So $f \circ g$ is the identity map on $B$.

Suppose $f$ is not a bijection. Then $f$ may not both be injective and surjective. If $f$ not injective, then from (2), $f$ does not have a right inverse, so there cannot exist a $g : B \rightarrow A$ where $f \circ g$ is the identity map on $B$. If $f$ is not surjective, then from (1) $f$ does not have a left inverse, so there cannot exist a $g : B \rightarrow A$ where $g \circ f$ is the identity map on $A$.

\item[(4)] Suppose $f : A \rightarrow B$ is bijective. Then by definition, $f$ is both injective and surjective.

Suppose $f$ is injective. We must show that $f$ is surjective. Let $n$ be the cardinality of $A$. If we index the elements of $A$, then $A = \left\lbrace a_1, a_2, ..., a_n \right\rbrace$. Because $f$ is injective, then $f(a_1) \neq f(a_2) \neq ... \neq f(a_n)$ - that is, $f$ maps precisely to $n$ elements of $B$. But there are $n$ elements of $B$, so the range of $f$ must necessarily be $B$. Therefore, $f$ is surjective.

Suppose $f$ is surjective. We must show that $f$ is injective. For every $b \in B$, the fiber of $f$ over $b$ has a cardinality of at least 1, and moreover each fiber of $f$ is disjoint. Let $\alpha$ be the set containing the elements of all fibers of $f$ over $b$. If there are $n$ elements of $B$, then the cardinality of $\alpha$ must be at least $n$. But there are $n$ elements of $A$, so the cardinality of $\alpha$ must be exactly $n$. In particular, $\alpha = A$. Furthermore, the cardinality of each fiber of $f$ is exactly 1, and since each fiber of $f$ is disjoint, then for any $a_1, a_2 \in \alpha$, $a_1$ and $a_2$ cannot be in the same fiber of $f$. Therefore, $f(a_1) \neq f(a_2)$, showing that $f$ is injective.
\end{itemize}

\subsection*{Proposition 2}
\begin{itemize}
\item[(1)] Suppose $\sim$ defines an equivalence relation on $A$. Since every element of $A$ is part of an equivalent class of $\sim$ and each equivalent class of $\sim$ cannot contain elements not in $A$, then the union of all equivalent classes of $\sim$ is equivalent to $A$. Now we must show that each equivalent class of $\sim$ is disjoint. Suppose for sake of contradiction that $a \in A$ is a member of at least two separate equivalent classes. Consider two such equivalent classes $\eta_1$ and $\eta_2$. Suppose without loss of generality that $a$ is the only equivalent class in $\eta_1$. But $\eta_1$ cannot be an equivalent class, since $\eta_1 \subseteq \eta_2$. Thus, suppose $\eta_1$ and $\eta_2$ have at least one additional element each: let $b \in \eta_1$ and $c \in \eta_2$. So $b \sim a$ and $a \sim c$. By the transitive property of equivalence relations, then $b \sim c$. But then $b$ and $c$ must be part of the same equivalent class. Therefore, $\eta_1$ and $\eta_2$ cannot be equivalent classes. By contradiction, $a$ can be a member of at most one equivalent class. Therefore, the set of equivalent classes of $\sim$ form a partition of $A$.
\item[(2)] Let $\left\lbrace A_i | i \in I \right\rbrace$ be a partition of $A$. Define a binary relation on $A$ as follows: for every $a, b \in A$, $a \sim b$ if $a, b \in A_i$ for some $i \in I$. If $a = b$, then clearly $a \sim b$. Additionally, it is clear that if $a \sim b$, then $b \sim a$. Further, if $a \sim b$ and $b \sim c$, then $a$ and $c$ must be in the same partition, so therefore $a \sim c$. Therefore, $\sim$ is an equivalence relation. Additionally, from the definition of $\sim$, then for some $a \in A$, the equivalence class of $a$ contains precisely the elements of $A$ that are part of the same partition of $a$. This completes the proof.
\end{itemize}
\subsection*{Exercises}
\begin{itemize}
\item[(1)]
$$X_1 = \begin{pmatrix}
1 & 1 \\
0 & 1
\end{pmatrix}$$
Since $X_1 = M$, then clearly $X_1 \in \mathcal{B}$
$$X_2 = \begin{pmatrix}
1 & 1 \\
1 & 1
\end{pmatrix}, X_2M = \begin{pmatrix}
1 & 2 \\
1 & 2
\end{pmatrix}, MX_2 = \begin{pmatrix}
2 & 2 \\
1 & 1
\end{pmatrix} \rightarrow X_2 \not \in \mathcal{B}$$
$$X_3 = \begin{pmatrix}
0 & 0 \\
0 & 0
\end{pmatrix}, X_3M = MX_3 = \begin{pmatrix}
0 & 0 \\
0 & 0
\end{pmatrix} \rightarrow X_3 \in \mathcal{B}$$
$$X_4 = \begin{pmatrix}
1 & 1 \\
1 & 0
\end{pmatrix}, X_4M = \begin{pmatrix}
1 & 2 \\
1 & 1
\end{pmatrix}, MX_4 = \begin{pmatrix}
2 & 1 \\
1 & 0
\end{pmatrix} \rightarrow X_4 \not \in \mathcal{B}$$
$$X_5 = \begin{pmatrix}
1 & 0 \\
0 & 1
\end{pmatrix}, X_5M = MX_5 = \begin{pmatrix}
1 & 1 \\
0 & 1
\end{pmatrix} \rightarrow X_5 \in \mathcal{B}$$
$$X_6 = \begin{pmatrix}
0 & 1 \\
1 & 0
\end{pmatrix}, X_6M = \begin{pmatrix}
1 & 1 \\
1 & 1
\end{pmatrix}, MX_6 = \begin{pmatrix}
1 & 1 \\
1 & 0
\end{pmatrix} \rightarrow X_6 \not \in \mathcal{B}$$
\item[(2)]
$$M(P + Q) = MP + MQ = PM + QM = (P + Q)M$$
Since $M(P + Q) = (P + Q)M$, then $P + Q \in \mathcal{B}$
\item[(3)]
$$M(PQ) = (MP)Q = (PM)Q = P(MQ) = P(QM) = (PQ)M$$
Since $MPQ = PQM$, then $PQ \in \mathcal{B}$
\item[(4)]
Let
$$X = \begin{pmatrix}
p & q \\
r & s
\end{pmatrix}$$
$$MX = \begin{pmatrix}
p + r & q + s \\
r & s
\end{pmatrix} = XM = \begin{pmatrix}
p & p + q \\
r & r + s
\end{pmatrix}$$
We can see that the following conditions need to be satisfied:
$$r = 0$$
$$p = s$$
Then, $X$ has the following form:
$$X = \begin{pmatrix}
p & q \\
0 & p
\end{pmatrix}$$
\item[(5)]
\begin{itemize}
\item[(a)] $f$ is not well defined. For example, $\frac{3}{2}$ has an additional rational representation $\frac{6}{4}$. It is ambiguous whether $f\left(\frac{3}{2}\right)$ either equals 3 or 6.
\item[(b)] $f$ is well defined. Note that if $\frac{a}{b} = \frac{c}{d}$, then $\frac{a^2}{b^2} = \frac{c^2}{d^2}$, so $f\left(\frac{a}{b}\right) = f\left(\frac{c}{d}\right) = \frac{c^2}{d^2}$
\end{itemize}
\item[(6)] $f$ is not well defined. Each integer has two real representations - ending in all 0s or ending in all 9s. For instance, the integer 1 has two representations: 1.000... or 0.999.... It is ambiguous then whether $f(1)$ either equals 0 or 9.
\item[(7)] First we must show that $\sim$ is an equivalence relation. \\
Reflexive: Since $f(a) = f(a)$, then $a \sim a$. \\
Symmetric: If $a \sim b$, then $f(a) = f(b)$, and $f(b) = f(a)$. Therefore, $b \sim a$. \\
Transitive. If $a \sim b$ and $b \sim c$, then $f(a) = f(b) = f(c)$. Therefore, $a \sim c$. \\
$\sim$ is therefore an equivalence relation. Now we must show that the equivalence classes of $\sim$ are precisely the fibers of $f$. Consider an equivalence class $X$ of $\sim$. Fix $x \in X$. For every element $a \in X$, $a \sim x$, and therefore $f(a) = b$ for a particular $b \in B$. So the equivalent class $X$ is the fiber of $f$ over $b$. Now consider the fiber $\alpha$ of $f$ over $b$. For all $a \in A$, we have $a \in \alpha$ if and only if $f(a) = f(x)$ for some $x \in X$. That is, $a \sim x$, and therefore $\alpha$ is an equivalence class of $sim$.
\end{itemize}

\section{Properties of the Integers}
\begin{itemize}
\item[(1)]
\begin{itemize}
\item[(a)]Applying the Euclidean Algorithm:
$$20 = (1)13 + 7$$
$$13 = (1)7 + 6$$
$$7 = (1)6 + 1$$
$$1 = (1)1$$
Thus the gcd of $20$ and $13$ is 1. The lcm is $(20 \cdot 13) / 1 = 260$. Now applying the generated equations in reverse order:
$$1 = (1)(7 - 6) = (1)7 - (1)6$$
$$1 = (1)7 - (1)(13 - 7) = (2)7 - (1)13$$
$$1 = (2)(20 - 13) - (1)13 = (2)20 - 3(13)$$
\item[(b)] Applying the Euclidean Algorithm:
$$372 = (5)69 + 27$$
$$69 = (2)27 + 15$$
$$27 = (1)15 + 12$$
$$15 = (1)12 + 3$$
$$12 = (4)3$$
Thus the gcd of $372$ and $69$ is 3. The lcm is $(372 \cdot 69) / 3 = 8556$. Now applying the generated equations in reverse order:
$$3 = 15 - (1)12$$
$$3 = 15 - (1)(27 - 15) = (2)15 + (-1)27$$
$$3 = (2)(69 - (2)27) + (-1)(27) = (2)69 + (-5)27$$
$$3 = (2)69 + (-5)(372 - (5)69) = (27)69 + (-5)372$$
\item[(c)] Applying the Euclidean Algorithm:
$$792 = (2)275 + 242$$
$$275 = (1)242 + 33$$
$$242 = (7)33 + 11$$
$$33 = (3)11$$
Thus the gcd of $792$ and $275$ is 11. The lcm is $(792 \cdot 275) / 11 = 19800$. Now applying the generated equations in reverse order:
$$11 = (1)242 - 7(33)$$
$$11 = (1)242 - 7(275 - 242) = (8)242 - (7)275$$
$$11 = (8)(792 - (2)275) - (7)275 = (8)792 - (23)275$$
\item[(d)] Applying the Euclidean Algorithm:
$$11391 = (2)5673 + 45$$
$$5673 = (126)45 + 3$$
$$45 = (15)3$$
Thus the gcd of $11391$ and $5673$ is 3. The lcm is $(11391 \cdot 5673) / 3 = 21540381$. Now applying the generated equations in reverse order:
$$3 = 5673 - (126)45$$
$$3 = 5673 - (126)(11391 - (2)5673) = (253)5673 - (126)11391$$
\item[(e)] Applying the Euclidean Algorithm:
$$1761 = (1)1567 + 194$$
$$1567 = (8)194 + 15$$
$$194 = (12)15 + 14$$
$$15 = (1)14 + 1$$
$$14 = (14)1$$
Thus the gcd of $1761$ and $1567$ is 1. The lcm is $(1761 \cdot 1567) / 1 = 2759487$. Now applying the generated equations in reverse order:
$$1 = 15 - 14$$
$$1 = 15 - (194 - (12)15) = (13)15 - 194$$
$$1 = (13)(1567 - (8)194) - 194 = (13)1567 - (105)194$$
$$1  (13)1567 - (105)(1761 - 1567) = (118)1567 - (105)1761$$
\item[(f)] Applying the Euclidean Algorithm:
$$507885 = (8)60808 + 21421$$
$$60808 = (2)21421 + 17966$$
$$21421 = (1)17966 + 3455$$
$$17966 = (5)3455 + 691$$
$$3455 = (5)691$$
Thus the gcd of $507885$ and $60808$ is 691. The lcm is $(507885 \cdot 60808) / 691 = 44693880$. Now applying the generated equations in reverse order:
$$691 = 17966 - (5)3455$$
$$691 = 17966 - (5)(21421 - 17966) = (6)17966 - (5)21421$$
$$691 = (6)(60808 - (2)21421) - (5)21421 = (6)(60808) - (17)21421$$
$$691 = (6)60808 - (17)(507885 - (8)60808)$$
$$691 = (142)60808 - (17)507885$$
\end{itemize}
\item[(2)]
If $k \in \mathbb{Z}$ divides the integers $a$ and $b$, then for some $m, n \in \mathbb{Z}$, $a = km$ and $b = kn$. Therefore,
$$as + bt = kms + knt = k(ms + nt) = kx$$
where $x = ms + nt$. Since $s, t \in \mathbb{Z}$, then $x = ms + nt \in \mathbb{Z}$. Therefore, $k$ must divide $as + bt$.
\item[(3)]
Suppose $n \in \mathbb{Z}$ is composite. Then $n = ab$ for some nonzero $a, b \in \mathbb{Z}$ such that $a, b \neq \pm 1$. Clearly, $n$ divides $ab$. Suppose for sake of contradiction without loss of generality that $n | a$. Then for some $x \in \mathbb{Z}$, $nx = a$. Therefore, we have $abx = a \rightarrow a(bx - 1) = 0 \rightarrow bx = 1 \rightarrow x = \frac{1}{b}$. But since $b \neq \pm 1$, then $x$ cannot be an integer. So by contradiction, $n \not | a$.
\item[(4)]
Let $x_0$ and $y_0$ be particular solutions to $ax + by = N$. We want to show that we also have the general solutions
$$x = x_0 + \frac{b}{d}t \text{ and } y = y_0 - \frac{a}{d}t$$
Substituting, we have
$$a\left(x_0 + \frac{b}{d}t\right) + b\left(y_0 - \frac{a}{d}t\right) = ax_0 + by_0 + \frac{ab}{d}t - \frac{ab}{d}t = ax_0 + by_0 = N$$
\item[(5)]
$$\varphi(1) = 1$$
$$\varphi(2) = 1$$
$$\varphi(3) = 2$$
$$\varphi(4) = 2$$
$$\varphi(5) = 4$$
$$\varphi(6) = \varphi(2)\varphi(3) = 2$$
$$\varphi(7) = 6$$
$$\varphi(8) = \varphi(2^3) = 2^2(2 - 1) = 4$$
$$\varphi(9) = \varphi(3^2) = 3(3 - 1) = 6$$
$$\varphi(10) = \varphi(2)\varphi(5) = 4$$
$$\varphi(11) = 10$$
$$\varphi(12) = \varphi(3)\varphi(4) = 4$$
$$\varphi(13) = 12$$
$$\varphi(14) = \varphi(2)\varphi(7) = 6$$
$$\varphi(15) = \varphi(3)\varphi(5) = 8$$
$$\varphi(16) = \varphi(2^4) = 2^3(2 - 1) = 8$$
$$\varphi(17) = 16$$
$$\varphi(18) = \varphi(2)\varphi(9) = 6$$
$$\varphi(19) = 18$$
$$\varphi(20) = \varphi(4)\varphi(5) = 8$$
$$\varphi(21) = \varphi(3)\varphi(7) = 12$$
$$\varphi(22) = \varphi(2)\varphi(11) = 10$$
$$\varphi(23) = 22$$
$$\varphi(24) = \varphi(3)\varphi(8) = 8$$
$$\varphi(25) = \varphi(5^2) = 5(5 - 1) = 20$$
$$\varphi(26) = \varphi(2)\varphi(13) = 12$$
$$\varphi(27) = \varphi(3^3) = 3^2(3 - 1) = 18$$
$$\varphi(28) = \varphi(4)\varphi(7) = 12$$
$$\varphi(29) = 28$$
$$\varphi(30) = \varphi(2)\varphi(15) = 8$$
\item[(6)] We shall proceed by means of induction. \\
Base Case: Let $A$ be a subset of $\mathbb{Z}^+$ of size 1. Clearly, $A$ has a minimal element - the single element of $A$. If $a$ is the element in $A$, then $a = a$. \\
Now suppose for all subsets of $\mathbb{Z}^+$ of size at most $k$, that each subset has a unique minimal element. Let $B$ be a subset of $\mathbb{Z}^+$ of size $k + 1$. We can partition $B$ into two subsets: $B_1$ and $B_k$. $B_k$ is a set of size $k$, and $B_1$ is a set of size 1. $B_k$ has a minimal element - call it $m$. Let $b$ be the element in $B_1$. If $m < b$, then $m$ is the minimal element of $B$. If $m > b$, then $b$ is the minimal element of $B$, since $b$ must necessarily be smaller than every element of $B_k$. Note that $m \neq b$ since $m$ and $b$ are in separate partitions of $B$, so therefore the minimal element must be unique.
\item[(7)] Suppose there exist nonzero integers $a$ and $b$ such that $a^2 = pb^2$, and that for sake of contradiction that $p$ is prime. By the fundamental theorem of arthemetic, we can express $a$ and $b$ as a product of distinct primes:
$$a = p_1^{\alpha_1}p_2^{\alpha_2}...p_n^{\alpha_n}, b = p_1^{\beta_1}p_2^{\beta_2}...p_n^{\beta_n}$$
where the exponents are nonnegative integers. Let $p = p_1$. Then
$$\frac{a^2}{b^2} = \frac{\left(p^{\alpha_1}p_2^{\alpha_2}p_3^{\alpha_3}...p_n^{\alpha_n} \right)^2}{\left(p^{\beta_1}p_2^{\beta_2}p_3^{\beta_3}...p_n^{\beta_n} \right)^2} = \frac{p^{2\alpha_1}p_2^{2\alpha_2}p_3^{2\alpha_3}...p_n^{2\alpha_n}}{p^{2\beta_1}p_2^{2\beta_2}p_3^{2\beta_3}...p_n^{2\beta_n}} = p$$
Then $alpha_2 = \alpha_3 = ... = \alpha_n = \beta_2 = \beta_3 = ... = \beta_n$ and $2\alpha_1 - 2\beta_1 = 1$ But then $\alpha_1 - \beta_1 = \frac{1}{2}$, contradicting that $\alpha_1$ and $\beta_1$ are integers. Therefore, $p$ must be composite.
\item[(8)]
Fix a prime $p$. Let $a$ be the highest power of $p$ that divides $n!$. That is, $p^a$ divides $n!$. Note that as $n$ increases to $n+1$, then $(n+1)!$ will have an additional factor $n+1$, which by the fundamental theorem of arithmetic can be decomposed into primes. If $p$ divides $n+1$, then additional factors of $p$ are added, increasing $a$. Furthermore, if $p^i$ divides $n+1$, for some $i > 0$, then at least $i$ factors of $p$ are added. In particular, if $p^i$ divides $n+1$, then $p$, $p^2$, ..., $p^{i-1}$ also divide $n+1$. So, we can simply count the number of times each power of $p$ divides into $1, 2, .., n+1$. In particular, the number of multiples of $p$ less than $n$ is $\left[ \frac{n}{p} \right]$ (this fact is a consequence of the Division Algorithm: $n = xp + r$ for $0 \leq r < p \rightarrow \frac{n}{p} = x + \frac{r}{p} \rightarrow \left[ \frac{n}{p} \right] = x$). Using these fact, we then have the following equation for obtaining the highest power of $p$ that divides $n!$:
$$a = \displaystyle\sum_{i=1}^\infty \left[\frac{n}{p^i} \right]$$
\item[(9)]
See 2-9.py
\item[(10)]
Recall that for any $n = p_1^{\alpha_1}p_2^{\alpha_2}...p_s^{\alpha_s}$, where $p_1, p_2, ..., p_s$ are primes and each exponent is $\geq 0$, that $$\varphi(n) = p_1^{\alpha_1 - 1}(p_1 - 1)p_2^{\alpha_2 - 1}(p_2 - 1)...p_s^{\alpha_s - 1}(p_s - 1)$$
Now we wish to determine for a fixed $N \in \mathbb{Z}^+$ whether there are finitely many integers $n$ such that $\varphi(n) = N$. We can see that for each prime $p$ that is a member of the prime factorization of $n$ (ie. if $p^\beta$ is the highest power of $p$ that divides into $n$, then $\beta > 0$), that $p < N$ from the equation we have recalled above. Since $N$ is finite, there are then a finite number of possible primes that can be members of the prime factorization of a particular $n$. Furthermore, there are a finite number of permutations of primes that can be used to generate all $n$ such that $\varphi(n) = N$. Therefore, there are finitely many integers $n$ such that $\varphi(n) = N$. As $n$ approaches infinity, the largest prime that can be a member of its prime factorization will also approach infinity. Therefore, $\varphi(n)$ also approaches infinity.
\item[(11)] Suppose $d$ divides $n$. Then for some $a \in \mathbb{Z}^+$, then $n = da$. Suppose $d$ and $a$ have the following prime factorizations:
$$d = p_1^{\alpha_1}p_2^{\alpha_2}...p_s^{\alpha_s}r_1^{a_1}r_2^{a_2}...r_x^{a_x}$$
$$a = q_1^{\beta_1}q_2^{\beta_2}...q_t^{\beta_t}r_1^{b_1}r_2^{b_2}...r_x^{b_x}$$
Ie. $r_1, r_2, ..., r_x$ are the common primes of the prime factorizations of $d$ and $a$.
Then $\varphi(n) = \varphi(da) = p_1^{\alpha_1 - 1}(p_1 - 1)p_2^{\alpha_2 - 1}(p_2 - 1)...p_s^{\alpha_s - 1}(p_s - 1)q_1^{\beta_1 - 1}(q_1 - 1)q_2^{\beta_2 - 1}(q_2 - 1)...q_t^{\beta_t - 1}(q_t - 1)r_1^{a_1+b_1-1}(r_1 - 1)r_2^{a_2+b_2 - 1}(r_2 - 1)...r_x^{a_x+b_x - 1}(r_x - 1)$. Note that $\varphi(d) = p_1^{\alpha_1 - 1}(p_1 - 1)p_2^{\alpha_2 - 1}(p_2 - 1)...p_s^{\alpha_s - 1}(p_s - 1)r_1^{a_1-1}(r_1 - 1)r_2^{a_2 - 1}(r_2 - 1)...r_x^{a_x - 1}(r_x - 1)$. Therefore, $\varphi(n) = \varphi(d)q_1^{\beta_1 - 1}(q_1 - 1)q_2^{\beta_2 - 1}(q_2 - 1)...q_t^{\beta_t - 1}(q_t - 1)r_1^{b_1}r_2^{b_2}...r_x^{b_x}$. Thus, $\varphi(d)$ divides $\varphi(n)$.
\end{itemize}

\end{document}