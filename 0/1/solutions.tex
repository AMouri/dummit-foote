\documentclass[12pt]{article}
\usepackage{amsmath}
\begin{document}
\title{Preliminaries - Basics}
\author{Alec Mouri}

\maketitle
\section*{Proposition 1}
\begin{itemize}
\item[(1)] Let $f$ be injective. Then for every $a \in A$, $f(a)$ is unique. Let $\beta = \left\lbrace f(a) \right\rbrace_{a \in A}$. Note that $\beta \subseteq B$. Define $g : B \rightarrow A$ as follows - for every $b \in \beta$ where $b = f(a)$ for some $a \in A$, then let $g(b) = a$. Moreover, for every $c \in B$ where $c \not \in \beta$, then $g(c)$ maps to some arbitrary element of $A$. $g \circ f$ is a mapping $A \rightarrow A$ where for $a \in A$, $(g \circ f)(a) = g(b) = a$. Therefore, $f$ has a left inverse $g$.

Suppose $f$ is not injective. Then for some two elements $a_1, a_2 \in A$, $f(a_1) = f(a_2) = b$. Clearly, $f$ does not have a left inverse - for every function $g : B \rightarrow A$, $g(b)$ can only map either to $a_1$ or $a_2$. Without loss of generality, consider $g(b) = a_1$. Then $(g \circ f)(a_2) = g(b) = a_1 \neq a_2$. Thus, $g \circ f$ is not the identity mapping on $A$.

\item[(2)] Let $f$ be surjective. Then for every $b \in B$, $\exists a \in A$ such that $f(a) = b$. That is, the fiber of $f$ over $b$ is nonempty. Define $h : B \rightarrow A$ as follows - for every $b \in B$, consider the fiber $\eta$ of $f$ over $b$. For some $a \in \eta$, then $h(b) = a$. $f \circ h$ is a mapping $B \rightarrow B$ where for $b \in B$, $(f \circ h)(b) = f(a) = b$. Therefore, $f$ has a right inverse $h$.

Suppose $f$ is not surjective. Then there exists some $b \in B$ where the fiber of $f$ over $b$ is empty. It is then clear that $f$ does not have a right inverse - $f \circ h$ cannot be the identity mapping on $B$ because although $b$ may be in the domain of $f \circ h$, $f$ may never output $b$.

\item[(3)] Let $f$ be a bijection. That is, $f$ is both injective and surjective. Then for every $a \in A$, $f(a)$ is unique. Moreover, for every $b \in B$, $\exists a \in A$ such that $f(a) = b$. Therefore, defne $g : B \rightarrow A$ as follows - for every $b \in B$ where $b = f(a)$ for some $a \in A$, then let $g(b) = a$. $g \circ f$ is a mapping $A \rightarrow A$ where for $a \in A$, $(g \circ f)(a) = g(b) = a$. So $g \circ f$ is the identity map on $A$. $f \circ g$ is a mapping $B \rightarrow B$ where for $b \in B$, $(f \circ g)(b) = f(a) = b$. So $f \circ g$ is the identity map on $B$.

Suppose $f$ is not a bijection. Then $f$ may not both be injective and surjective. If $f$ not injective, then from (2), $f$ does not have a right inverse, so there cannot exist a $g : B \rightarrow A$ where $f \circ g$ is the identity map on $B$. If $f$ is not surjective, then from (1) $f$ does not have a left inverse, so there cannot exist a $g : B \rightarrow A$ where $g \circ f$ is the identity map on $A$.

\item[(4)] Suppose $f : A \rightarrow B$ is bijective. Then by definition, $f$ is both injective and surjective.

Suppose $f$ is injective. We must show that $f$ is surjective. Let $n$ be the cardinality of $A$. If we index the elements of $A$, then $A = \left\lbrace a_1, a_2, ..., a_n \right\rbrace$. Because $f$ is injective, then $f(a_1) \neq f(a_2) \neq ... \neq f(a_n)$ - that is, $f$ maps precisely to $n$ elements of $B$. But there are $n$ elements of $B$, so the range of $f$ must necessarily be $B$. Therefore, $f$ is surjective.

Suppose $f$ is surjective. We must show that $f$ is injective. For every $b \in B$, the fiber of $f$ over $b$ has a cardinality of at least 1, and moreover each fiber of $f$ is disjoint. Let $\alpha$ be the set containing the elements of all fibers of $f$ over $b$. If there are $n$ elements of $B$, then the cardinality of $\alpha$ must be at least $n$. But there are $n$ elements of $A$, so the cardinality of $\alpha$ must be exactly $n$. In particular, $\alpha = A$. Furthermore, the cardinality of each fiber of $f$ is exactly 1, and since each fiber of $f$ is disjoint, then for any $a_1, a_2 \in \alpha$, $a_1$ and $a_2$ cannot be in the same fiber of $f$. Therefore, $f(a_1) \neq f(a_2)$, showing that $f$ is injective.
\end{itemize}

\section*{Proposition 2}
\begin{itemize}
\item[(1)] Suppose $\sim$ defines an equivalence relation on $A$. Since every element of $A$ is part of an equivalent class of $\sim$ and each equivalent class of $\sim$ cannot contain elements not in $A$, then the union of all equivalent classes of $\sim$ is equivalent to $A$. Now we must show that each equivalent class of $\sim$ is disjoint. Suppose for sake of contradiction that $a \in A$ is a member of at least two separate equivalent classes. Consider two such equivalent classes $\eta_1$ and $\eta_2$. Suppose without loss of generality that $a$ is the only equivalent class in $\eta_1$. But $\eta_1$ cannot be an equivalent class, since $\eta_1 \subseteq \eta_2$. Thus, suppose $\eta_1$ and $\eta_2$ have at least one additional element each: let $b \in \eta_1$ and $c \in \eta_2$. So $b \sim a$ and $a \sim c$. By the transitive property of equivalence relations, then $b \sim c$. But then $b$ and $c$ must be part of the same equivalent class. Therefore, $\eta_1$ and $\eta_2$ cannot be equivalent classes. By contradiction, $a$ can be a member of at most one equivalent class. Therefore, the set of equivalent classes of $\sim$ form a partition of $A$.
\item[(2)] Let $\left\lbrace A_i | i \in I \right\rbrace$ be a partition of $A$. Define a binary relation on $A$ as follows: for every $a, b \in A$, $a \sim b$ if $a, b \in A_i$ for some $i \in I$. If $a = b$, then clearly $a \sim b$. Additionally, it is clear that if $a \sim b$, then $b \sim a$. Further, if $a \sim b$ and $b \sim c$, then $a$ and $c$ must be in the same partition, so therefore $a \sim c$. Therefore, $\sim$ is an equivalence relation. Additionally, from the definition of $\sim$, then for some $a \in A$, the equivalence class of $a$ contains precisely the elements of $A$ that are part of the same partition of $a$. This completes the proof.
\end{itemize}
\section*{Exercises}
\begin{itemize}
\item[(1)]
$$X_1 = \begin{pmatrix}
1 & 1 \\
0 & 1
\end{pmatrix}$$
Since $X_1 = M$, then clearly $X_1 \in \mathcal{B}$
$$X_2 = \begin{pmatrix}
1 & 1 \\
1 & 1
\end{pmatrix}, X_2M = \begin{pmatrix}
1 & 2 \\
1 & 2
\end{pmatrix}, MX_2 = \begin{pmatrix}
2 & 2 \\
1 & 1
\end{pmatrix} \rightarrow X_2 \not \in \mathcal{B}$$
$$X_3 = \begin{pmatrix}
0 & 0 \\
0 & 0
\end{pmatrix}, X_3M = MX_3 = \begin{pmatrix}
0 & 0 \\
0 & 0
\end{pmatrix} \rightarrow X_3 \in \mathcal{B}$$
$$X_4 = \begin{pmatrix}
1 & 1 \\
1 & 0
\end{pmatrix}, X_4M = \begin{pmatrix}
1 & 2 \\
1 & 1
\end{pmatrix}, MX_4 = \begin{pmatrix}
2 & 1 \\
1 & 0
\end{pmatrix} \rightarrow X_4 \not \in \mathcal{B}$$
$$X_5 = \begin{pmatrix}
1 & 0 \\
0 & 1
\end{pmatrix}, X_5M = MX_5 = \begin{pmatrix}
1 & 1 \\
0 & 1
\end{pmatrix} \rightarrow X_5 \in \mathcal{B}$$
$$X_6 = \begin{pmatrix}
0 & 1 \\
1 & 0
\end{pmatrix}, X_6M = \begin{pmatrix}
1 & 1 \\
1 & 1
\end{pmatrix}, MX_6 = \begin{pmatrix}
1 & 1 \\
1 & 0
\end{pmatrix} \rightarrow X_6 \not \in \mathcal{B}$$
\item[(2)]
$$M(P + Q) = MP + MQ = PM + QM = (P + Q)M$$
Since $M(P + Q) = (P + Q)M$, then $P + Q \in \mathcal{B}$
\item[(3)]
$$M(PQ) = (MP)Q = (PM)Q = P(MQ) = P(QM) = (PQ)M$$
Since $MPQ = PQM$, then $PQ \in \mathcal{B}$
\item[(4)]
Let
$$X = \begin{pmatrix}
p & q \\
r & s
\end{pmatrix}$$
$$MX = \begin{pmatrix}
p + r & q + s \\
r & s
\end{pmatrix} = XM = \begin{pmatrix}
p & p + q \\
r & r + s
\end{pmatrix}$$
We can see that the following conditions need to be satisfied:
$$r = 0$$
$$p = s$$
Then, $X$ has the following form:
$$X = \begin{pmatrix}
p & q \\
0 & p
\end{pmatrix}$$
\item[(5)]
\begin{itemize}
\item[(a)] $f$ is not well defined. For example, $\frac{3}{2}$ has an additional rational representation $\frac{6}{4}$. It is ambiguous whether $f\left(\frac{3}{2}\right)$ either equals 3 or 6.
\item[(b)] $f$ is well defined. Note that if $\frac{a}{b} = \frac{c}{d}$, then $\frac{a^2}{b^2} = \frac{c^2}{d^2}$, so $f\left(\frac{a}{b}\right) = f\left(\frac{c}{d}\right) = \frac{c^2}{d^2}$
\end{itemize}
\item[(6)] $f$ is not well defined. Each integer has two real representations - ending in all 0s or ending in all 9s. For instance, the integer 1 has two representations: 1.000... or 0.999.... It is ambiguous then whether $f(1)$ either equals 0 or 9.
\item[(7)] First we must show that $\sim$ is an equivalence relation. \\
Reflexive: Since $f(a) = f(a)$, then $a \sim a$. \\
Symmetric: If $a \sim b$, then $f(a) = f(b)$, and $f(b) = f(a)$. Therefore, $b \sim a$. \\
Transitive. If $a \sim b$ and $b \sim c$, then $f(a) = f(b) = f(c)$. Therefore, $a \sim c$. \\
$\sim$ is therefore an equivalence relation. Now we must show that the equivalence classes of $\sim$ are precisely the fibers of $f$. Consider an equivalence class $X$ of $\sim$. Fix $x \in X$. For every element $a \in X$, $a \sim x$, and therefore $f(a) = b$ for a particular $b \in B$. So the equivalent class $X$ is the fiber of $f$ over $b$. Now consider the fiber $\alpha$ of $f$ over $b$. For all $a \in A$, we have $a \in \alpha$ if and only if $f(a) = f(x)$ for some $x \in X$. That is, $a \sim x$, and therefore $\alpha$ is an equivalence class of $sim$.
\end{itemize}

\end{document}