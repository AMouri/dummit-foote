\documentclass[12pt]{article}
\begin{document}
\title{Subgroups - Centralizers and Normalizers, Stabilizers and Kernels}
\author{Alec Mouri}

\maketitle
\section*{Exercises}
\begin{itemize}
\item[(1)]
Let $\mathcal{H} = \{ g \in G | g^{-1}ag = a, \forall a \in A \}$. Suppose $h \in H$. Then for all $a \in A$, 
$$h^{-1}ah = a \rightarrow ah = ha \rightarrow a = hah^{-1}$$
Thus, $h \in C_G(A)$, and $\mathcal{H} \subseteq C_G(A)$. Now suppose $c \in C_G(A)$. Then for all $a \in A$,
$$cac^{-1} = a \rightarrow ca = ac \rightarrow a = c^{-1}ac$$
Thus, $c \in \mathcal{H}$, and $C_G(A) \subseteq \mathcal{H}$. Thus, $C_G(A) = \mathcal{H}$.
\item[(2)]
Since $C_G(A) \subseteq G$ for any subset $A$ of $G$, then $C_G(Z(G)) \subseteq G$. Suppose $g \in G$, and let $a \in Z(G)$. Then, $ag = ga \rightarrow g^{-1}ag$. Thus, $G \subseteq C_G(Z(G))$, so therefore $C_G(Z(G)) = G$.

Since $C_G(Z(G)) \leq N_G(Z(G))$, and $N_G(Z(G)) \leq G$, then $N_G(Z(G)) = G$.
\end{itemize}
\end{document}