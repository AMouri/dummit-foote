\documentclass[12pt]{article}
\usepackage{amssymb}
\begin{document}
\title{Subgroups - Definition and Examples}
\author{Alec Mouri}

\maketitle
\section*{Exercises}
\begin{itemize}
\item[(1)]
Let $H$ be the specified subset of the group $G$.
\begin{itemize}
\item[(a)]
Since $0 \in H$, then $H \neq \emptyset$. Let $x = a + ai, x = b + bi$. Clearly, $A, B \in H$. And, $-b - bi = y^{-1}$. Then, 
$$xy^{-1} = a + ai + -b - bi = (a-b) + (a-b)i \in H$$
Thus, by the Subgroup Criterion $H \leq G$
\item[(b)]
Since $1 \in H$, then $H \neq \emptyset$. Let $x = a + bi$, $y = c + di$, where $x, y \in H$. In particular, $a^2 + b^2 = c^2 + d^2 = 1$. Note that $y^{-1} = (c^2 + d^2)^{-1}(c - di) = c - di$. Then,
$$xy^{-1} = (a + bi)(c - di) = ac + bd + (bc - ad)i$$
Since 
$$(ac + bd)^2 + (bc - ad)^2 = a^2c^2 + 2abcd + b^2d^2 + b^2c^2 - 2abcd + a^2d^2$$ 
$$= a^2(c^2 + d^2) + b^2(c^2 + d^2) = a^2 + b^2 = 1$$
Then $xy^{-1} \in H$. Thus, by the Subgroup Criterion $H \leq G$
\item[(c)]
Note that $0 = \frac{0}{n} \in H$, so $H \neq \emptyset$. Suppose $p, q | n$. If $n = \alpha_1^{\beta_1}\alpha_2^{\beta_2}...\alpha_k^{\beta_k}$, where each $\alpha_i$ is a distinct prime, and $\beta_i \geq 1$, then $p = \alpha_1^{\delta_1}\alpha_2^{\delta_2}...\alpha_k^{\delta_k}$ and $q = \alpha_1^{\gamma_1}\alpha_2^{\gamma_2}...\alpha_k^{\gamma_k}$, where for each $i$, $0 \leq \delta_i, \gamma_i \leq \beta_i$. Let $x = \frac{a}{p}, y = \frac{b}{q} \in H$. Note that $y^{-1} = \frac{-b}{q}$. Let $g = gcd(p, q)$. Then $p = gc$ and $q = gd$. Let $\ell = lcm(p, q)$, so that $g\ell = pq$. Then
$$xy^{-1} =  \frac{a}{p} + \frac{-b}{q} = \frac{aq - bp}{pq} = \frac{g(ac - bd)}{g\ell} = \frac{ac - bd}{\ell}$$

Clearly, $\ell | n$, so therefore $xy^{-1} \in H$. Thus, by the Subgroup Criterion, $H \leq G$.
\item[(d)]
Note that $0 = \frac{0}{n} \in H$, so $H \neq \emptyset$. Suppose for $p, q \in \mathbb{Z}^+$, $gcd(p,n) = gcd(q, n) = 1$. Note that for $w, x, y, z \in \mathbb{Z}$, then $wp + xn = yq + zn = 1$. Then
$$1 = (wp + xn)(yq + zn) = wypq + xynq + wznp + xzn^2$$
$$= pq(wy) + n(xyq + wzp + xzn) \rightarrow gcd(pq, n) = 1$$
Let $x = \frac{a}{p}, y = \frac{b}{q} \in H$. Note that $y^{-1} = \frac{-b}{q}$. Then,
$$xy^{-1} = \frac{a}{p} + \frac{-b}{q} = \frac{aq - bp}{pq} \in H$$
Thus, by the Subgroup Criterion, $H \leq G$. 
\item[(e)]
Note that $1^2 = \frac{1}{1} \rightarrow 1 \in H$, so $H \neq \emptyset$. Let $a, b \in H$, where $a^2 = \frac{w}{x}$ and $b^2 = \frac{y}{z}$. Note that $b^{-1} = b\frac{z}{y}$, and $(ab)^2 = a^2b^2 = \frac{wy}{xz} \in H$, so
$$(ab^{-1})^2 = (ab\frac{z}{y})^2 = \frac{wz}{xy} \rightarrow ab^{-1} \in H$$
Thus, by the Subgroup Criterion, $H \leq G$.
\end{itemize}
\item[(2)]
Let $H$ be the specified subset of the group $G$.
\begin{itemize}
\item[(a)]
Consider $x = (1 \, 2) \in H$. Then $xx^{-1} = 1$, But, $1 \not \in H$, so $H$ is not a subgroup of $G$.
\item[(b)]
Consider $x = s \in H$. Then $ss^{-1} = 1$, But, $1 \not \in H$, so $H$ is not a subgroup of $G$.
\item[(c)]
Consider $x \in H$. Then $x^n = 1$, where $n = ab$. But then $(x^a)^b = 1$, so $x^a$ has order at most $b$, and $x^a \not \in H$. Thus $H$ is not a subgroup of $G$.
\item[(d)]
Note that $1 \in H$, but $1 + 1 = 2 \not \in H$, so $H$ is not closed under addition. Thus, $H$ is not a subgroup of $G$.
\item[(e)]
Consider $\sqrt{2}$ and $\sqrt{3}$. Then 
$$(\sqrt{2} - \sqrt{3})^2 = 2 - 2\sqrt{6} + 3 = 5 - 2\sqrt{6}$$
Since $\sqrt{2} - \sqrt{3} \not \in H$, then $H$ is not a subgroup of $G$.
\end{itemize}
\item[(3)]
\begin{itemize}
\item[(a)]
Let $H = \left\lbrace s^ar^b | 0 \leq a \leq 1, b \in \left\lbrace 0, 2 \right\rbrace \right\rbrace$. Clearly, $H \neq \emptyset$. $x = s^ar^b, y = s^cr^d \in H$. Th`en
$$xy^{-1} = s^ar^br^{-d}s^{-c} = s^ar^{b - d}s^{-c} = s^{a-c}r^{d-b}$$
If $a = c$, then $s^{a-c} = 1$. If $a > c$, then $s^{a-c} = s$. If $a < c$, then $s^{a-c} = s^{-1} = s$.

If $d = b$, then $r^{d-b} = 1$. If $d > b$, then $r^{d - b} = r^2$. If $d < b$, then $r^{d - b} = r^{-2} = r^2$.

Thus, $xy^{-1} \in H$, so by the Subgroup Criterion $H \leq D_8$.
\item[(b)]
Let $H = \left\lbrace s^ar^b | 0 \leq a \leq 1, b \in \left\lbrace a, a + 2 \right\rbrace \right\rbrace$. Clearly, $H \neq \emptyset$. $x = s^ar^b, y = s^cr^d \in H$. Then
$$xy^{-1} = s^ar^br^{-d}s^{-c} = s^ar^{b - d}s^{-c} = s^{a-c}r^{d-b}$$
If $a = 0, c = 0, b = 0, d = 0$, then $xy^{-1} = 1$

If $a = 0, c = 0, b = 2, d = 0$, then $xy^{-1} = r^2$

If $a = 0, c = 0, b = 0, d = 2$, then $xy^{-1} = r^2$

If $a = 0, c = 0, b = 2, d = 2$, then $xy^{-1} = 1$

If $a = 1, c = 0, b = 1, d = 0$, then $xy^{-1} = sr^3$

If $a = 1, c = 0, b = 3, d = 0$, then $xy^{-1} = sr$

If $a = 1, c = 0, b = 1, d = 2$, then $xy^{-1} = sr$

If $a = 1, c = 0, b = 3, d = 2$, then $xy^{-1} = sr^3$

If $a = 0, c = 1, b = 0, d = 1$, then $xy^{-1} = sr$

If $a = 0, c = 1, b = 2, d = 1$, then $xy^{-1} = sr^3$

If $a = 0, c = 1, b = 0, d = 3$, then $xy^{-1} = sr^3$

If $a = 0, c = 1, b = 2, d = 3$, then $xy^{-1} = sr$

If $a = 1, c = 1, b = 1, d = 1$, then $xy^{-1} = 1$

If $a = 1, c = 1, b = 3, d = 1$, then $xy^{-1} = r^2$

If $a = 1, c = 1, b = 1, d = 3$, then $xy^{-1} = r^2$

If $a = 1, c = 1, b = 3, d = 3$, then $xy^{-1} = 1$


Thus, $xy^{-1} \in H$, so by the Subgroup Criterion $H \leq D_8$.
\end{itemize}
\item[(4)]
Consider $\mathbb{Z}$ and $\mathbb{Z}^+$, but under addition. Clearly, $\mathbb{Z}$ is a group, and $\mathbb{Z}^+ \subset \mathbb{Z}$, and $|\mathbb{Z}^+| = \infty$. Furthermore, for $a, b \in \mathbb{Z}^+$, then $a + b > a$ and $a + b > b$, so $a + b \in \mathbb{Z}^+$, and $\mathbb{Z}^+$ is closed under addition. But, since $0 \not \in \mathbb{Z}^+$, then $\mathbb{Z}^+$ is not a subgroup of $\mathbb{Z}$.
\item[(5)]
Suppose $H$ is a subgroup of $G$ where $|H| = n - 1$. Then there exists precisely one $\alpha \in G$ such that $\alpha \not \in H$. Let $x \in H$ be a nonidentity element. If $x\alpha = y \in H$, then $\alpha = yx^{-1}$. But since $H$ is a subgroup, then $\alpha \in H$, a contradiction. If $x\alpha \not \in H$, then $x\alpha = \alpha \rightarrow x = 1$, another contradiction. Therefore, $H$ cannot exist.
\item[(6)]
Let $H = \left\lbrace g \in G | |g| < \infty \right\rbrace$. Since $1 \in H$, then $H \neq \emptyset$. Let $h_1, h_2 \in H$. Since $H$ is abelian, then
$$(h_1h_2^{-1})^{|h_1||h_2|} = h_1^{|h_1||h_2|}h_2^{-|h_1||h_2|} (h_1^{|h_1|})^{|h_2|}(h_2^{|h_2|})^{-|h_1|} = 1$$
Thus, $h_1h_2^{-1} \in H$, so therefore $H \leq G$.

Consider the infinite dihedral group $D_\infty$: $\left\lbrace r, s | s^2 = 1, rs = sr^{-1} \right\rbrace$. Let $H = \left\lbrace g \in D_\infty | |D_\infty| < \infty \right\rbrace$. Observe that $D_\infty$ is non-abelian. Note that $s^2 = 1$, and $(rs)^2 = rsrs = rr^{-1}s^2 = 1$, so $s, rs \in H$. But, $(rs)s = r$. Since $r$ has infinite order, then $r \not \in H$, so $H$ is not a subgroup of $G$.
\item[(7)]
Note that for $a \in \mathbb{Z}$, then $|a| = \infty$ if and only if $a \neq 0$. And for $b \in \mathbb{Z}/n\mathbb{Z}$, then $|b| \leq n$. Thus, the torsion subgroup of $\mathbb{Z} \times (\mathbb{Z}/n\mathbb{Z})$ is $\left\lbrace (0, a) | a \in \mathbb{Z}/n\mathbb{Z} \right\rbrace$.

Note that $(1, 1) + (-1, 1) = (0, 2)$ is not in the set of elements of infinite order with the identity. Therefore, the set is not closed under addition, and therefore is not a subgroup.
\item[(8)]
Suppose $H \cup K$ is a subgroup. Suppose for sake of contradiction that $H \cap K \neq H$ and $H \cap K \neq K$. Then we have $h \in H$ where $h \not \in K$, and $k \in K$ where $k \not \in H$. Then either $hk \in H$ or $hk \in K$. If $h' = hk \in H$, then $k = h^{-1}h' \in H$, a contradiction. Similarly, if $k' = hk \in K$, then $h = k'k^{-1} \in K$, a contradiction. Thus, either $H \cap K \subseteq H$ or $H \cap K \subseteq K$. Ie. either $H \subseteq K$ or $K \subseteq H$.

Suppose without loss of generality that $H \subseteq K$. Then $H \cup K = K$, so therefore $H \cup K$ is a subgroup.
\item[(9)]
Since $I \in SL_n(F)$, then $SL_n(F) \neq \emptyset$. For $A, B \in SL_n(F)$, $\det(AB^{-1}) = \det(A)\det(B^{-1}) = \det(A)\det(B)^{-1} = 1$. Thus, $AB^{-1} \in SL_n(F)$. Thus by the subgroup criterion, $SL_n(F) \leq GL_n(F)$.
\item[(10)]
\begin{itemize}
\item[(a)]
Note that $1 \in H \cap K$ since $1 \in H$ and $1 \in K$, so $H \cap K \neq \emptyset$. Consider $a, b \in H \cap K$. For $a, b \in H \cap K$, note that $a, b \in H$ and $a, b \in K$. Then $ab^{-1} \in H$ and $ab^{-1} \in K$. Thus, $ab^{-1} \in H \cap K$. Therefore by the Subgroup Criterion, $H \cap K$ is a subgroup of $G$.
\item[(b)]
Let $\mathcal{A}$ be the collection of subgroups of $G$. Let $\mathcal{B}$ be the intersection of all elements of $\mathcal{A}$. Note that for each $a \in \mathcal{A}$, $1 \in \mathcal{A}$, so therefore $1 \in \mathcal{B}$. For $x, y \in \mathcal{B}$, note that for $a \in \mathcal{A}$, $x, y \in a$. Then $xy^{-1} \in a$. Thus, $ab^{-1} \in \mathcal{B}$. Therefore by the Subgroup Criterion, $\mathcal{B}$ is a subgroup of $G$.
\end{itemize}
\item[(11)]
\begin{itemize}
\item[(a)]
Let $H = \{ (a, 1) | a \in A \}$. Note that $(1, 1) \in H$ is the identity of $H$, so $H \neq \emptyset$. Let $a, b \in A$, so $(a, 1), (b, 1) \in H$. Note that $(b, 1)^{-1} = (b^{-1} , 1)$, so $(a, 1)(b^{-1}, 1) = (ab^{-1}, 1) \in H$. Thus $H$ is a subgroup of $A \times B$.
\item[(b)]
Let $H = \{ (1, b) | b \in B \}$. Note that $(1, 1) \in H$ is the identity of $H$, so $H \neq \emptyset$. Let $a, b \in B$, so $(1, a), (1, b) \in H$. Note that $(1, b)^{-1} = (1, b^{-1})$, so $(1, a)(1, b^{-1}) = (1, ab^{-1}) \in H$. Thus $H$ is a subgroup of $A \times B$.
\item[(c)]
Let $H = \{ (a, a) | a \in A \}$. Note that $(1, 1) \in H$ is the identity of $H$, so $H \neq \emptyset$. Let $a, b \in A$, so $(a, a), (b, b) \in H$. Note that $(b, b)^{-1} = (b^{-1} , b^{-1})$, so $(a, a)(b^{-1}, b^{-1}) = (ab^{-1}, ab^{-1}) \in H$. Thus $H$ is a subgroup of $A \times B$.
\end{itemize}
\item[(12)]
\begin{itemize}
\item[(a)]
Let $H = \{ a^n | a \in A \}$. Since $1^n = 1 \in H$, then $H \neq \emptyset$. Let $a, b \in A$, so $a^n, b^n \in H$. Then $a^n(b^n)^{-1} = a^n(b^{-1})^n = (ab^{-1})^n \in H$. Thus, $H$ is a subgroup of $A$.
\item[(b)]
Let $H = \{ a \in A | a^n = 1 \}$. Since $1^n = 1$, then $1 \in H$ and $H \neq \emptyset$. Let $a, b \in H$. Then $(ab^{-1})^n = a^n(b^{n})^{-1} = 1$, so $ab^{-1} \in H$.  Thus, $H$ is a subgroup of $A$.
\end{itemize}
\item[(13)] 
Note that clearly $H \in 0$. Suppose $H \neq 0$. Let $\frac{p}{q} \in H$ such that $gcd(p, q) = 1$. Note that $\frac{p}{q}q = p \in H$, and $\frac{1}{p} \in H$. Thus, $p\frac{1}{p} = 1 \in H$, and therefore for each $n \in \mathbb{Z}$, $n \in H$. Then for each $n$, $\frac{1}{n} \in H \rightarrow \frac{m}{n} \in H$ for each $m, n \in \mathbb{Z}$. Thus, $H = \mathbb{Q}$.
\item[(14)]
Let $H = \{ x \in D_{2n} | x^2 = 1 \}$. Note that since $s^2 = 1$ and $(sr)^2 = 1$, so $s, sr \in H$. But, $ssr = r \not \in H$, so $H$ is not closed under the group operation. Thus $H$ is not a subgroup of $D_{2n}$.
\item[(15)]
Note that $\cup_{i=1}^\infty H_i \neq \emptyset$. Let $x, y \in \cup_{i=1}^\infty H_i$. Then for some $1 \leq a \leq b < \infty$, we can say without loss of generality that $x \in H_a, y \in H_b$. Then $x \in H_b$. So, $xy^{-1} \in H_b$, and $xy^{-1} \in \cup_{i=1}^\infty H_i$. Thus by the Subgroup Criterion, $\cup_{i=1}^\infty H_i \leq G$
\item[(16)]
Let $H$ be the set of $n \times n$ upper triangular matrices. Note that $I \in H$, so $H \neq \emptyset$. We shall now prove a few lemmas:

Lemma 1: Let $A_n \in H$. Then $\det(A) = \prod_{i}^n a_{ii}$. In particular, the diagonal entries $a_{ii}$ of $A$ are all nonzero. Proof: If $n = 1$, then $\det(A_1) = n_{11}$, so the lemma is true. Suppose the lemma is true for $n - 1$. Then, 
$$\det(A_n) = \sum_{k=1}^{n}(-1)^{n + k}a_{n k}\det(D_{n, k})$$
where $D_{n, k}$ is the resulting matrix from deleting the $n$th row and the $k$th column. Since $a_{n k} = 0$ for $k < n$, then $\det(A_n) = (-1)^{2n}a_{nn}\det(D_{n, k}) = a_{nn}\det(D_{n, k})$. Since $D_{n, k}$ is upper triangular, then $D_{n, k} = \prod_{i=1}^{n-1}a_{ii}$. Thus, $\det(A_n) = \prod_{i=1}^n a_{ii}$.

Lemma 2: If $A, B \in H$, then $AB \in H$. Proof: Let $C = AB$. Then for $i, j$, $c_{ij} = \sum_{x = 1}^{n} a_{ix}b_{xj}$. Suppose $i > j$. Then for $i < x$, then $a_{ix} = 0$, and for $j > x$, then $b_{xj} = 0$. Thus, $c_{ij} = 0$. Suppose $i = j$. Then $c_{ij} = \sum_{x=1}^n a_{ix}b_{xi}$ For $i < x$, then $a_{ix} = 0$, and for $i > x$, then $b_{xi} = 0$, so $c_{ij} = a_{ii}b_{ii} \neq 0$. Thus, $\det(AB) \neq 0$, $AB$ is upper triangular, and therefore $AB \in H$.

Lemma 3: If $A \in H$, then $B = A^{-1} \in H$. Proof: 
It sufficies to show that $B_{ij} = 0$ for $i > j$, and $B_{ii} \neq 0$ for all $i$. Let $i = n$. Then for all $j$, $0 = (AB)_{nj} = \sum_{k=1}^n A_{nk}B_{kj} = A_{nn}B_{kj} \rightarrow B_{kj} = 0$: the last implication is from Lemma 1. Now suppose for some $i + 1$, $B_{i+1, j} = 0$ for $i+1 > j$. Then for all $j$, $0 = (AB)_{ij} = \sum_{k=1}^n A_{ik}B_{kj} = \sum_{k=1}^i A_{ik}B_{kj} = A_{ii}B_{ki} \rightarrow B_{ki} = 0$. Furthermore, for all $i$, $1 = (AB)_{ii} = \sum_{k=1}^n A_{ik}B_{ki} = \sum_{k=1}^n \sum_{k=1}^i A_{ik}B_{ki} = A_{ii}B_{ii} \rightarrow B_{ii} = A_{ii}^{-1} \neq 0$. Thus, $B \in H$.

Let $A, B \in H$. Then by Lemmas 2 and 3, $AB^{-1} \in H$. Thus by the Subgroup Criterion, $H \leq GL_n(F)$. 
\end{itemize}
\end{document}