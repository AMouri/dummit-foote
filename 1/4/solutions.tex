\documentclass[12pt]{article}
\usepackage{amsmath, amssymb}
\begin{document}
\title{Introduction to Groups - Matrix Groups}
\author{Alec Mouri}

\maketitle
\section*{Exercises}
\begin{itemize}
\item[(1)]
Consider a $2 \times 2$ matrix
$$A = \begin{pmatrix}
a & b \\
c & d
\end{pmatrix}$$
where $a, b, c, d \in \mathbb{F}_2$. $A \in GL_2(\mathbb{F}_2)$ if $\det(A) = ad - bc \neq 0$. Ie. either $ad = 0$ and $bc = 1$, or $ad = 1$ and $bc = 0$. There are then 6 possible sets of values for $a, b, c, d$:
$$a = 0, b = 1, c = 1, d = 0$$
$$a = 1, b = 1, c = 1, d = 0$$
$$a = 0, b = 1, c = 1, d = 1$$
$$a = 1, b = 0, c = 0, d = 1$$
$$a = 1, b = 1, c = 0, d = 1$$
$$a = 1, b = 0, c = 1, d = 1$$
Thus, $|GL_2(\mathbb{F}_2)| = 6$.
\item[(2)]
\begin{itemize}
\item[1.]
$$\begin{pmatrix}
0 & 1 \\
1 & 0
\end{pmatrix}^2 = \begin{pmatrix}
1 & 0 \\
0 & 1
\end{pmatrix} \rightarrow \left| \begin{pmatrix}
0 & 1 \\
1 & 0
\end{pmatrix} \right| = 2$$
\item[2.]
$$\begin{pmatrix}
1 & 1 \\
1 & 0
\end{pmatrix}^2 = \begin{pmatrix}
0 & 1 \\
1 & 1
\end{pmatrix}$$
$$\begin{pmatrix}
1 & 1 \\
1 & 0
\end{pmatrix}^3 = \begin{pmatrix}
1 & 0 \\
0 & 1
\end{pmatrix} \rightarrow \left| \begin{pmatrix}
1 & 1 \\
1 & 0
\end{pmatrix} \right| = 3$$
\item[3.]
$$\begin{pmatrix}
0 & 1 \\
1 & 1
\end{pmatrix}^2 = \begin{pmatrix}
1 & 1 \\
1 & 0
\end{pmatrix}$$
$$\begin{pmatrix}
0 & 1 \\
1 & 1
\end{pmatrix}^3 = \begin{pmatrix}
1 & 0 \\
0 & 1
\end{pmatrix} \rightarrow \left| \begin{pmatrix}
0 & 1 \\
1 & 1
\end{pmatrix} \right| = 3$$
\item[4.]
$$\left| \begin{pmatrix}
1 & 0 \\
0 & 1
\end{pmatrix} \right| = 1$$
\item[5.]
$$\begin{pmatrix}
1 & 1 \\
0 & 1
\end{pmatrix}^2 = \begin{pmatrix}
1 & 0 \\
0 & 1
\end{pmatrix} \rightarrow \left| \begin{pmatrix}
1 & 1 \\
0 & 1
\end{pmatrix} \right| = 2$$
\item[6.]
$$\begin{pmatrix}
1 & 0 \\
1 & 1
\end{pmatrix}^2 = \begin{pmatrix}
1 & 0 \\
0 & 1
\end{pmatrix} \rightarrow \left| \begin{pmatrix}
1 & 0 \\
1 & 1
\end{pmatrix} \right| = 2$$
\end{itemize}
\item[(3)]
Let
$$a = \begin{pmatrix}
0 & 1 \\
1 & 1
\end{pmatrix}, b = \begin{pmatrix}
1 & 1 \\
0 & 1
\end{pmatrix}$$
Then
$$ab = \begin{pmatrix}
0 & 1 \\
1 & 1
\end{pmatrix}\begin{pmatrix}
1 & 1 \\
0 & 1
\end{pmatrix} = \begin{pmatrix}
0 & 1 \\
1 & 0
\end{pmatrix}$$
But
$$ba = \begin{pmatrix}
1 & 1 \\
0 & 1
\end{pmatrix}\begin{pmatrix}
0 & 1 \\
1 & 1
\end{pmatrix} = \begin{pmatrix}
1 & 0 \\
1 & 1
\end{pmatrix}$$
Since $ab \neq ba$, then $GL_2(\mathbb{F}_2)$ is non-abelian.
\item[(4)]
If $n$ is not prime, then for some prime $p$, then $n = pk$ for $k \in \mathbb{Z}^+$. From Chapter 0, Section 3, Exercise 12, we know that there cannot be an integer $c$ such that $pc \equiv 1 \mod n$. Therefore, $p$ does not have a multiplicative inverse in $\mathbb{Z}/n\mathbb{Z}$, therefore $\mathbb{Z}/n\mathbb{Z}$ is not a field.
\item[(5)]
Suppose $F$ has an infinite number of elements. For each $a \in F$, let $A = aI$, where $I$ is the $n \times n$ identity matrix. Since $\det(A) = a^n$, then $A \in GL_n(F)$. There are infinitely many such matrices, so $GL_n(F)$ is an infinite group.

Suppose $F$ has a finite number of elements. For an $n \times n$ matrix, there are $n^2$ elements of the matrix. Therefore there are $|F|^{n^2}$ $n \times n$ matrices. Since $GL_n(F)$ is a subset of $n \times n$ matrices, then $GL_n(F)$ is a finite group.
\item[(6)]
For an $n \times n$ matrix, there are $n^2$ elements of the matrix. Therefore there are $q^{n^2}$ $n \times n$ matrices. Since $GL_n(F)$ is a subset of $n \times n$ matrices, then $|GL_n(F)| \leq q^{n^2}$. Since the zero matrix is not in $GL_n(F)$, then the inequality is strict.
\item[(7)]
The total number of $2 \times 2$ matrices over $\mathbb{F}_p$ is $p^4$. Consider an non-invertible matrix. Then one row is a multiple of the other. For a particular row where at least one entry is nonzero, there are $p$ rows that are multiples of that entry. There are $p^2 - 1$ such rows. If both entries are nonzero, then $p^2$ rows are multiples of that row. Therefore, there are $p(p^2 - 1) + p^2 = p^3 + p^2 - p$. Therefore,
$$|GL_2(\mathbb{F}_p)| = p^4 - (p^3 + p^2 - p) = p^4 - p^3 - p^2 + p$$
\item[(8)]
First, note that every field contains the additive identity 0, and the multiplicative identity 1. Furthermore, also note that $0 \neq 1$. Consider two $n \times n$ matrices $A$ and $B$, where
$$A = \begin{pmatrix}
1 & 1 & 0 & \cdots & 0 \\
0 & 1 & 0 & \cdots & 0 \\
0 & 0 & 1 & \cdots & 0 \\
\vdots & \vdots & \vdots & \ddots & \vdots \\
0 & 0 & 0 & \cdots & 1
\end{pmatrix}, B = \begin{pmatrix}
1 & 0 & 0 & \cdots & 0 \\
1 & 1 & 0 & \cdots & 0 \\
0 & 0 & 1 & \cdots & 0 \\
\vdots & \vdots & \vdots & \ddots & \vdots \\
0 & 0 & 0 & \cdots & 1
\end{pmatrix}$$
Ie., $A$ is the identity matrix with entry $(1, 2)$ is a 1 instead of a 0, and $B = A^\top$. Note that $\det(A) = \det(B) = 1$, so $A, B \in GL_n(F)$. Then, entry $(1,1)$ of $AB$ is $1 + 1 = 2$, and entry $(1,1)$ of $BA$ is 1. If $2 = 1$, then $1 + 1 = 1 + 0 \rightarrow 1 = 0$, a contradiction. Therefore, $2 \neq 1$, therefore $AB \neq BA$, and $GL_n(F)$ is nonabelian.
\item[(9)]Consider matrices $A, B, C$:
$$A = \begin{pmatrix}
a_{11} & a_{12} \\
a_{21} & a_{22}
\end{pmatrix}, B = \begin{pmatrix}
b_{11} & b_{12} \\
b_{21} & b_{22}
\end{pmatrix}, C = \begin{pmatrix}
c_{11} & c_{12} \\
c_{21} & c_{22}
\end{pmatrix}$$
Then
\tiny
$$(AB)C = \left( \begin{pmatrix}
a_{11} & a_{12} \\
a_{21} & a_{22}
\end{pmatrix}\begin{pmatrix}
b_{11} & b_{12} \\
b_{21} & b_{22}
\end{pmatrix} \right)\begin{pmatrix}
c_{11} & c_{12} \\
c_{21} & c_{22}
\end{pmatrix}$$
$$= \begin{pmatrix}
a_{11}b_{11} + a_{12}b_{21} & a_{11}b_{12} + a_{12}b_{22} \\
a_{21}b_{11} + a_{22}b_{21} & a_{21}b_{12} + a_{22}b_{22}
\end{pmatrix}\begin{pmatrix}
c_{11} & c_{12} \\
c_{21} & c_{22}
\end{pmatrix}$$
$$= \begin{pmatrix}
c_{11}(a_{11}b_{11} + a_{12}b_{21}) + c_{21}(a_{11}b_{12} + a_{12}b_{22}) & c_{12}(a_{11}b_{11} + a_{12}b_{21}) + c_{22}(a_{11}b_{12} + a_{12}b_{22}) \\
c_{11}(a_{21}b_{11} + a_{22}b_{21}) +  c_{21}(a_{21}b_{12} + a_{22}b_{22}) & c_{12}(a_{21}b_{11} + a_{22}b_{21}) +  c_{22}(a_{21}b_{12} + a_{22}b_{22})
\end{pmatrix}$$
$$= \begin{pmatrix}
a_{11}(b_{11}c_{11} + b_{12}c_{21}) + a_{12}(b_{12}c_{11} + b_{22}c_{21}) & a_{11}(b_{11}c_{12} + b_{12}c_{22}) + a_{12}(b_{21}c_{12} + b_{22}c_{22}) \\
a_{21}(b_{11}c_{11} + b_{12}c_{21}) + a_{22}(b_{21}c_{11} + b_{22}c_{21}) & a_{21}(b_{11}c_{12} + b_{12}c_{22}) + a_{22}(b_{21}c_{12} + b_{22}c_{22})
\end{pmatrix}$$
$$= \begin{pmatrix}
a_{11} & a_{12} \\
a_{21} & a_{22}
\end{pmatrix}\begin{pmatrix}
b_{11}c_{11} + b_{12}c_{21} & b_{21}c_{11} + b_{22}c_{21}\\
b_{12}c_{11} + b_{22}c_{21} & b_{21}c_{12} + b_{22}c_{22}
\end{pmatrix}$$
$$= \begin{pmatrix}
a_{11} & a_{12} \\
a_{21} & a_{22}
\end{pmatrix} \left (\begin{pmatrix}
b_{11} & b_{12} \\
b_{21} & b_{22}
\end{pmatrix}\begin{pmatrix}
c_{11} & c_{12} \\
c_{21} & c_{22}
\end{pmatrix} \right) = A(BC)$$
\normalsize
\item[(10)]
\begin{itemize}
\item[(a)]
$$\mathcal{A} = \begin{pmatrix}
a_1 & b_1 \\
0 & c_1
\end{pmatrix}\begin{pmatrix}
a_2 & b_2 \\
0 & c_2
\end{pmatrix} = \begin{pmatrix}
a_1a_2 & a_1b_2 + b_1c_2 \\
0 & c_1c_2
\end{pmatrix}$$
Since $a_1, a_2, c_1, c_2 \neq 0$, then $a_1a_2, c_1c_2 \neq 0$ ($\mathbb{R} - \left\lbrace 0 \right\rbrace$ is closed under multiplication). Therefore, $\mathcal{A} \in G$, and $G$ is closed under multiplication.
\item[(b)]
Suppose we have matrices $\mathcal{A}, \mathcal{A}^{-1} \in G$ such that
$$\mathcal{A} = \begin{pmatrix}
a & b \\
0 & c
\end{pmatrix}, \mathcal{A}^{-1} = \begin{pmatrix}
x & y \\
0 & z
\end{pmatrix}, \mathcal{A}\mathcal{A}^{-1} = I$$
Then
$$\begin{pmatrix}
ax & ay + bz \\
0 & cz
\end{pmatrix} = \begin{pmatrix}
1 & 0 \\
0 & 1
\end{pmatrix}$$
Thus, we have $x = a^{-1}, y = -a^{-1}bc^{-1}, z = c^{-1}$. Since $\mathbb{R} - \left\lbrace 0 \right\rbrace$ is closed under inverses, then $x, z \neq 0$, and $\mathcal{A}^{-1} \in G$. Therefore, $G$ is closed under inverses
\item[(c)]
Since $G \subseteq GL_2(\mathbb{R})$ (the determinant of any matrix in $G$ is nonzero), contains the identity $I$, is closed under matrix multiplication (part a), and there exists an inverse for each element (part b), then $G$ is a subgroup of $GL_2(\mathbb{R})$.
\item[(d)]
Let $G'$ be the set of elements of $G$ whose two diagonal entries are equal. Clearly, $I \in G'$. Since
$$\begin{pmatrix}
a_1 & b_1 \\
0 & a_1
\end{pmatrix}\begin{pmatrix}
a_2 & b_2 \\
0 & a_2
\end{pmatrix} = \begin{pmatrix}
a_1a_2 & a_1b_2 + a_2b_1 \\
0 & a_1a_2
\end{pmatrix} \in G'$$
Then $G'$ is closed under matrix multiplication. Furthermore, since for $a \neq 0$ and $b$,
$$\begin{pmatrix}
a & b \\
0 & a
\end{pmatrix}, \begin{pmatrix}
a^{-1} -a^{-1}ba^{-1} \\
0 & a^{-1}
\end{pmatrix} \in G'$$
And,
$$\begin{pmatrix}
a & b \\
0 & a
\end{pmatrix}\begin{pmatrix}
a^{-1} & -a^{-1}ba^{-1} \\
0 & a^{-1}
\end{pmatrix} = \begin{pmatrix}
1 & 0 \\
0 & 1
\end{pmatrix} = I$$
Then $G'$ is closed under inverses. Therefore, $G'$ is a subgroup of $G$, and therefore is a subgroup of $GL_2(\mathbb{R})$.
\end{itemize}
\item[(11)]
\begin{itemize}
\item[(a)]
$$XY = \begin{pmatrix}
1 & a & b \\
0 & 1 & c \\
0 & 0 & 1
\end{pmatrix}\begin{pmatrix}
1 & d & e \\
0 & 1 & f \\
0 & 0 & 1
\end{pmatrix} = \begin{pmatrix}
1 & d + a & e + af + b \\
0 & 1 & f + c \\
0 & 0 & 1
\end{pmatrix} \in H(F)$$
Thus $H(F)$ is closed under matrix multiplication. Suppose $a = b = c = d = e = 1$, and $f = 2$. Then
$$XY = \begin{pmatrix}
1 & 1 & 1 \\
0 & 1 & 1 \\
0 & 0 & 1
\end{pmatrix}\begin{pmatrix}
1 & 1 & 1 \\
0 & 1 & 2 \\
0 & 0 & 1
\end{pmatrix} = \begin{pmatrix}
1 & 2 & 4 \\
0 & 1 & 3 \\
0 & 0 & 1
\end{pmatrix}$$
But
$$YX = \begin{pmatrix}
1 & 1 & 1 \\
0 & 1 & 2 \\
0 & 0 & 1
\end{pmatrix}\begin{pmatrix}
1 & 1 & 1 \\
0 & 1 & 1 \\
0 & 0 & 1
\end{pmatrix} = \begin{pmatrix}
1 & 2 & 3 \\
0 & 1 & 3 \\
0 & 0 & 1
\end{pmatrix}$$
$XY \neq YX$, therefore $H(F)$ is non-abelian.
\item[(b)]
Let $Y$ be the inverse of $X$. Then
$$\begin{pmatrix}
1 & d + a & e + af + b \\
0 & 1 & f + c \\
0 & 0 & 
\end{pmatrix} = I$$.
Therefore, $d = -a, e = ac - b, f = -c$. So,
$$X^{-1} = \begin{pmatrix}
1 & -a & ac - b \\
0 & 1 & -c \\
0 & 0 & 1
\end{pmatrix} \in H(F)$$
Thus, $H(F)$ is closed under inverses.
\item[(c)]
Let
$$A = \begin{pmatrix}
1 & a_1 & a_2 \\
0 & 1 & a_3 \\
0 & 0 & 1
\end{pmatrix}, B = \begin{pmatrix}
1 & b_1 & b_2 \\
0 & 1 & b_3 \\
0 & 0 & 1
\end{pmatrix}, C = \begin{pmatrix}
1 & c_1 & c_2 \\
0 & 1 & c_3 \\
0 & 0 & 1
\end{pmatrix}$$
Then,
$$(AB)C = \left( \begin{pmatrix}
1 & a_1 & a_2 \\
0 & 1 & a_3 \\
0 & 0 & 1
\end{pmatrix}\begin{pmatrix}
1 & b_1 & b_2 \\
0 & 1 & b_3 \\
0 & 0 & 1
\end{pmatrix} \right) \begin{pmatrix}
1 & c_1 & c_2 \\
0 & 1 & c_3 \\
0 & 0 & 1
\end{pmatrix}$$
$$= \begin{pmatrix}
1 & a_1 + b_1 & a_2 + a_1b_3 + b_2 \\
0 & 1 & a_3 + b_3 \\
0 & 0 & 1
\end{pmatrix}\begin{pmatrix}
1 & c_1 & c_2 \\
0 & 1 & c_3 \\
0 & 0 & 1
\end{pmatrix}$$
$$= \begin{pmatrix}
1 & a_1 + b_1 + c_1 & c_2 + a_1c_3 + b_1c_3 + a_2 + a_1b_3 + b_2 \\
0 & 1 & a_3 + b_3 + c_3 \\
0 & 0 & 1
\end{pmatrix}$$
$$= \begin{pmatrix}
1 & a_1 & a_2 \\
0 & 1 & a_3 \\
0 & 0 & 1
\end{pmatrix}\begin{pmatrix}
1 & b_1 + c_1 & b_2 + b_1c_3 + c_2 \\
0 & 1 & b_3 + c_3 \\
0 & 0 & 1
\end{pmatrix}$$
$$= \begin{pmatrix}
1 & a_1 & a_2 \\
0 & 1 & a_3 \\
0 & 0 & 1
\end{pmatrix} \left(
\begin{pmatrix}
1 & b_1 & b_2 \\
0 & 1 & b_3 \\
0 & 0 & 1
\end{pmatrix}\begin{pmatrix}
1 & c_1 & c_2 \\
0 & 1 & c_3 \\
0 & 0 & 1
\end{pmatrix} \right) = A(BC)$$
Therefore, matrix multiplication is associative under $H(F)$. For each $a, b, c$, there are $|F|$ possibilities. Therefore, $|H(F)| = |F|^3$.
\item[(d)]
$$\left| \begin{pmatrix}
1 & 0 & 0 \\
0 & 1 & 0 \\
0 & 0 & 1
\end{pmatrix} \right| = 1$$
$$\begin{pmatrix}
1 & 1 & 0 \\
0 & 1 & 0 \\
0 & 0 & 1
\end{pmatrix}^2 = \begin{pmatrix}
1 & 0 & 0 \\
0 & 1 & 0 \\
0 & 0 & 1
\end{pmatrix} \rightarrow \left| \begin{pmatrix}
1 & 1 & 0 \\
0 & 1 & 0 \\
0 & 0 & 1
\end{pmatrix} \right| = 2$$
$$\begin{pmatrix}
1 & 0 & 1 \\
0 & 1 & 0 \\
0 & 0 & 1
\end{pmatrix}^2 = \begin{pmatrix}
1 & 0 & 0 \\
0 & 1 & 0 \\
0 & 0 & 1
\end{pmatrix} \rightarrow \left| \begin{pmatrix}
1 & 0 & 1 \\
0 & 1 & 0 \\
0 & 0 & 1
\end{pmatrix} \right| = 2$$
$$\begin{pmatrix}
1 & 0 & 0 \\
0 & 1 & 1 \\
0 & 0 & 1
\end{pmatrix}^2 = \begin{pmatrix}
1 & 0 & 0 \\
0 & 1 & 0 \\
0 & 0 & 1
\end{pmatrix} \rightarrow \left| \begin{pmatrix}
1 & 0 & 0 \\
0 & 1 & 1 \\
0 & 0 & 1
\end{pmatrix} \right| = 2$$
$$\begin{pmatrix}
1 & 1 & 1 \\
0 & 1 & 0 \\
0 & 0 & 1
\end{pmatrix}^2 = \begin{pmatrix}
1 & 0 & 0 \\
0 & 1 & 0 \\
0 & 0 & 1
\end{pmatrix} \rightarrow \left| \begin{pmatrix}
1 & 1 & 1 \\
0 & 1 & 0 \\
0 & 0 & 1
\end{pmatrix} \right| = 2$$
$$\begin{pmatrix}
1 & 0 & 1 \\
0 & 1 & 1 \\
0 & 0 & 1
\end{pmatrix}^2 = \begin{pmatrix}
1 & 0 & 0 \\
0 & 1 & 0 \\
0 & 0 & 1
\end{pmatrix} \rightarrow \left| \begin{pmatrix}
1 & 0 & 1 \\
0 & 1 & 1 \\
0 & 0 & 1
\end{pmatrix} \right| = 2$$
$$\begin{pmatrix}
1 & 1 & 0 \\
0 & 1 & 1 \\
0 & 0 & 1
\end{pmatrix}^2 = \begin{pmatrix}
1 & 0 & 1 \\
0 & 1 & 0 \\
0 & 0 & 1
\end{pmatrix}$$
$$\begin{pmatrix}
1 & 1 & 0 \\
0 & 1 & 1 \\
0 & 0 & 1
\end{pmatrix}^3 = \begin{pmatrix}
1 & 1 & 1 \\
0 & 1 & 1 \\
0 & 0 & 1
\end{pmatrix}$$
$$\begin{pmatrix}
1 & 1 & 0 \\
0 & 1 & 1 \\
0 & 0 & 1
\end{pmatrix}^4 = \begin{pmatrix}
1 & 0 & 0 \\
0 & 1 & 0 \\
0 & 0 & 1
\end{pmatrix} \rightarrow \left| \begin{pmatrix}
1 & 1 & 0 \\
0 & 1 & 1 \\
0 & 0 & 1
\end{pmatrix} \right| = 4$$
$$\begin{pmatrix}
1 & 1 & 1 \\
0 & 1 & 1 \\
0 & 0 & 1
\end{pmatrix}^2 = \begin{pmatrix}
1 & 0 & 1 \\
0 & 1 & 0 \\
0 & 0 & 1
\end{pmatrix}$$
$$\begin{pmatrix}
1 & 1 & 1 \\
0 & 1 & 1 \\
0 & 0 & 1
\end{pmatrix}^3 = \begin{pmatrix}
1 & 1 & 0 \\
0 & 1 & 1 \\
0 & 0 & 1
\end{pmatrix}$$
$$\begin{pmatrix}
1 & 1 & 1 \\
0 & 1 & 1 \\
0 & 0 & 1
\end{pmatrix}^4 = \begin{pmatrix}
1 & 0 & 0 \\
0 & 1 & 0 \\
0 & 0 & 1
\end{pmatrix} \rightarrow \left| \begin{pmatrix}
1 & 1 & 1 \\
0 & 1 & 1 \\
0 & 0 & 1
\end{pmatrix} \right| = 4$$
\item[(e)]
Claim: For any $n \geq 1$ and
$$X = \begin{pmatrix}
1 & a & b \\
0 & 1 & c \\
0 & 0 & 1
\end{pmatrix}, X^n = \begin{pmatrix}
1 & na & nb + ac\sum_{i = 0}^{n - 1} i \\
0 & 1 & nc \\
0 & 0 & 1
\end{pmatrix}$$
Base Case:
$$X^1 = \begin{pmatrix}
1 & a & b \\
0 & 1 & c \\
0 & 0 & 1
\end{pmatrix} = X$$
Suppose our claim is true for $n - 1$. Then,
$$X^{n} = X^{n-1}X = \begin{pmatrix}
1 & (n - 1)a & (n - 1)b + ac\sum_{i = 0}^{n - 2} i \\
0 & 1 & (n - 1)c \\
0 & 0 & 1
\end{pmatrix}\begin{pmatrix}
1 & a & b \\
0 & 1 & c \\
0 & 0 & 1
\end{pmatrix}$$
$$= \begin{pmatrix}
1 & (n-1)a + a & b + (n - 1)ac + (n - 1)b + ac\sum_{i = 0}^{n - 2} i \\
0 & 1 & (n - 1)c + c \\
0 & 0 & 1
\end{pmatrix}$$
$$= \begin{pmatrix}
1 & na & nb + ac\sum_{i=1}^{n-1}i \\
0 & 1 & nc \\
0 & 0 & 1
\end{pmatrix}$$
Thus our claim is true. Note that for $n \geq 1$, $X^n \neq I$ unless $X = I$. Therefore, every nonidentity element of $H(F)$ has infinite order.
\end{itemize}
\end{itemize}
\end{document}