\documentclass[12pt]{article}
\begin{document}
\title{Introduction to Groups - The Quaternion Group}
\author{Alec Mouri}

\maketitle
\section*{Exercises}
\begin{itemize}
\item[(1)]
$$|1| = 1$$
$$(-1)^2 = 1 \rightarrow |-1| = 1$$
$$i^2 = -1, i^3 = -i, i^4 = 1 \rightarrow |i| = 4$$
$$(-i)^2 = -1, (-i)^3 = i, (-i)^4 = 1 \rightarrow |-i| = 4$$
$$j^2 = -1, j^3 = -j, j^4 = 1 \rightarrow |j| = 4$$
$$(-j)^2 = -1, (-j)^3 = j, (-j)^4 = 1 \rightarrow |-j| = 4$$
$$k^2 = -1, k^3 = -k, k^4 = 1 \rightarrow |k| = 4$$
$$(-k)^2 = -1, (-k)^3 = k, (-k)^4 = 1 \rightarrow |-k| = 4$$
\item[(2)]
\begin{itemize}
\item[$S_3:$]
\begin{tabular}{| l || c | c | c | c | c | r |}
\hline
& 1 & (1 2) & (1 3) & (2 3) & (1 2 3) & (1 3 2) \\
\hline
\hline
1 & 1 & (1 2) & (1 3) & (2 3) & (1 2 3) & (1 3 2) \\
\hline
(1 2) & (1 2) & 1 & (1 2 3) & (1 3 2) & (1 3) & (2 3) \\
\hline
(1 3) & (1 3) & (1 3 2) & 1 & (1 2 3) & (2 3) & (1 2) \\
\hline
(2 3) & (2 3) & (1 2 3) & (1 3 2) & 1 & (1 2) & (1 3) \\
\hline
(1 2 3) & (1 2 3) & (2 3) & (1 2) & (1 3) & (1 3 2) & 1 \\
\hline
(1 3 2) & (1 3 2) & (1 3) & (2 3) & (1 2) & 1 & (1 2 3) \\
\hline
\end{tabular}
\item[$D_8$:]
\begin{tabular}{| l || c | c | c | c | c | c | c | r |}
\hline
& 1 & $r$ & $r^2$ & $r^3$ & $s$ & $sr$ & $sr^2$ & $sr^3$ \\
\hline
\hline
1 & 1 & $r$ & $r^2$ & $r^3$ & $s$ & $sr$ & $sr^2$ & $sr^3$ \\
\hline
$r$ & $r$ & $r^2$ & $r^3$ & 1 & $sr^3$ & $s$ & $sr$ & $sr^2$ \\
\hline
$r^2$ & $r^2$ & $r^3$ & 1 & $r$ & $sr^2$ & $sr^3$ & $s$ & $sr$ \\
\hline
$r^3$ & $r^3$ & 1 & $r$ & $r^2$ & $sr$ & $sr^2$ & $sr^3$ & $s$ \\
\hline
$s$ & $s$ & $sr$ & $sr^2$ & $sr^3$ & 1 & $r$ & $r^2$ & $r^3$ \\
\hline
$sr$ & $sr$ & $sr^2$ & $sr^3$ & $s$ & $r^3$ & 1 & $r$ & $r^2$ \\
\hline
$sr^2$ & $sr^2$ & $sr^3$ & $s$ & $sr$ & $r^2$ & $r^3$ & 1 & $r$ \\
\hline
$sr^3$ & $sr^3$ & $s$ & $sr$ & $sr^2$ & $r$ & $r^2$ & $r^3$ & 1 \\
\hline
\end{tabular}
\item[$Q_3:$]
\begin{tabular}{| l || c | c | c | c | c | c | c | r |}
\hline
& 1 & -1 & $i$ & $-i$ & $j$ & $-j$ & $k$ & $-k$ \\
\hline
\hline
1 & 1 & -1 & $i$ & $-i$ & $j$ & $-j$ & $k$ & $-k$ \\
\hline
-1 & -1 & 1 & $-i$ & $i$ & $-j$ & $j$ & $-k$ & $k$ \\
\hline
$i$ & $i$ & $-i$ & -1 & 1 & $k$ & $-k$ & $-j$ & $j$ \\
\hline
$-i$ & $-i$ & $i$ & 1 & -1 & $-k$ & $k$ & $j$ & $-j$ \\
\hline
$j$ & $j$ & $-j$ & $-k$ & $k$ & -1 & 1 & $i$ & $-i$ \\
\hline
$-j$ & $-j$ & $j$ & $k$ & $-k$ & 1 & -1 & $-i$ & $i$ \\
\hline
$k$ & $k$ & $-k$ & $j$ & $-j$ & $-i$ & $i$ & -1 & 1 \\
\hline
$-k$ & $-k$ & $k$ & $-j$ & $j$ & $i$ & $-i$ & 1 & -1 \\
\hline
\end{tabular}
\end{itemize}
\item[(3)]
Consider the elements $i, j$. Then
$$1 = i^4 = j^4$$
$$-1 = i^2 = j^2$$
$$i = i$$
$$-i = i^3$$
$$j = j$$
$$-j = j^3$$
$$k = ij$$
$$-k = ji$$
We then have the relations:
$$i^2 = j^2, i^4 = j^4 = 1, ij = i^2ji$$
Therefore, we have
$Q_8 = <x , y | x^2 = y^2, x^4 = y^4 = 1, xy = y^2xy >$, where $x = i$ and $y = j$.
\end{itemize}
\end{document}