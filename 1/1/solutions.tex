\documentclass[12pt]{article}
\usepackage{amsmath, amssymb}
\begin{document}
\title{Introduction to Groups}
\author{Alec Mouri}

\maketitle
\section{Basic Axioms and Examples}
\subsection*{Exercises}
\begin{itemize}
\item[(1)]
\begin{itemize}
\item[(a)] No. Let $a = 2, b = 3, c + 4$.
$$(2 - 3) - 4 = -5$$
$$2 - (3 - 4) = 3$$
\item[(b)] Yes. Consider $a, b, c \in \mathbb{R}$
$$a * b = a + b + ab$$
$$(a * b) * c = a + b + ab + c + (a + b + ab)c = a + b + c + ab + ac + bc + abc$$
$$= a + (b + c + bc) + a(b + c + bc) = a * (b * c)$$
\item[(c)] No. Let $a = 1, b = 2, c = 3$
$$(a * b) * c = \frac{\frac{1 + 2}{5} + 3}{5} = \frac{18}{25}$$
$$a * (b * c) = \frac{1 + \frac{2 + 3}{5}}{5} = \frac{10}{25}$$
\item[(d)] Yes. Consider $(a, b), (c, d), (e, f) \in \mathbb{Z} \times \mathbb{Z}$.
$$(a, b) * (c, d) = (ad + bc, bd)$$
$$((a, b) * (c, d)) * (e, f) = ((ad + bc)f + bde, bdf) = (adf + bcf + bde, bdf)$$
$$= (a(df) + b(cf + de), b(df)) = (a, b) * ((c, d) * (e, f))$$
\item[(e)] No. Let $a = 2, b = 3, c = 4$.
$$(a * b) * c = \frac{\frac{2}{3}}{4} = \frac{1}{6}$$
$$a * (b * c) = \frac{2}{\frac{3}{4}} = \frac{8}{3}$$
\end{itemize}
\item[(2)]
\begin{itemize}
\item[(a)] No. Let $a = 1$ and $b = 2$.
$$a * b = 1 - 2 = -1$$
$$b * a = 2 - 1 = 1$$
\item[(b)]Yes.
$$a * b = a + b + ab = b + a + ba = b * a$$
\item[(c)] Yes.
$$a * b = \frac{a + b}{5} = \frac{b + a}{5} = b * a$$
\item[(d)] Yes.
$$(a, b) * (c, d) = (ad + bc, bd) = (cb + da, db) = (c, d) * (a, b)$$
\item[(e)] No. Let $a = 2$ and $b = 3$
$$a * b = \frac{2}{3}$$
$$b * a = \frac{3}{2}$$
\end{itemize}
\item[(3)]
$$(\overline{a} + \overline{b}) + \overline{c} = \overline{a + b} + \overline{c} = \overline{a + b + c} = \overline{a} + \overline{b + c} = \overline{a} + (\overline{b} + \overline{c})$$
\item[(4)]
$$(\overline{a} \cdot \overline{b})\overline{c} = \overline{ab} \cdot \overline{c} = \overline{abc} = \overline{a} \cdot \overline{bc} = \overline{a}(\overline{b} \cdot \overline{c})$$
\item[(5)]
Consider 0 $\in \mathbb{Z}/n\mathbb{Z}$. Suppose 0 is an identity of $\mathbb{Z}/n\mathbb{Z}$. Then $a0 \equiv a \mod n$. But this is clearly false for $a \not \equiv 0 \mod n$. Then, 0 cannot be an identity of $\mathbb{Z}/n\mathbb{Z}$. But then for all $a \in \mathbb{Z}/n\mathbb{Z}$, $a0 \equiv 0 \mod n$, implying that the inverse of 0 does not exist. Therefore, $mathbb{Z}/n\mathbb{Z}$ is not a group under multiplication of residue classes.
\item[(6)]
For brevity, let $\mathcal{A}$ be the proposed group in question.
\begin{itemize}
\item[(a)]
Consider $\frac{a}{b}, \frac{c}{d}, \frac{e}{f} \in \mathcal{A}$. Then
$$\frac{a}{b} + \frac{c}{d} = \frac{ad + bc}{bd}$$
Note that since $b, d$ are odd, then $bd$ has no factors of 2 in its prime decomposition. Therefore, $\frac{ad + bc}{bd}$ will have an odd denominator if reduced to lowest terms. So, addition is a binary operation over $\mathcal{A}$. Furthermore, 
$$\left(\frac{a}{b} + \frac{c}{d} \right) + \frac{e}{f} = \frac{ad + bc}{bd} + \frac{e}{f} = \frac{adf + bcf + bde}{bdf}$$
$$ = \frac{a}{b} + \frac{cf + bd}{df} = \frac{a}{b} + \left(\frac{c}{d} + \frac{b}{f}\right)$$
Therefore, addition is associative over $\mathcal{A}$. We also have $\frac{0}{1} + \frac{a}{b} = \frac{a}{b} = \frac{0}{1}$, so $\frac{0}{1}$ is an identity under $\mathcal{A}$. And, $\frac{a}{b} + \frac{-a}{b} = 0 = \frac{-a}{b} + \frac{a}{b}$, so $\frac{-a}{b}$ is an inverse of $\frac{a}{b}$. Thus $\mathcal{A}$ is a group under addition.
\item[(b)]$\mathcal{A}$ is not a group under addition. Consider $a = \frac{1}{2}$ and $b = \frac{1}{6}$. Then
$$a + b = \frac{1}{2} + \frac{1}{6} = \frac{2}{3} \not \in \mathcal{A}$$
Thus, addition is not a binary operation over $\mathcal{A}$.
\item[(c)]$\mathcal{A}$ is not a group under addition. Consider $a = \frac{1}{2}$ and $b = \frac{2}{3}$. Then
$$a + b = \frac{1}{2} + \frac{2}{3} = \frac{7}{6} \not \in \mathcal{A}$$
Thus, addition is not a binary operation over $\mathcal{A}$.
\item[(d)]
$\mathcal{A}$ is not a group under addition. Consider $a = \frac{3}{2}$ and $b = \frac{-1}{1}$. Then
$$a + b = \frac{3}{2} + \frac{-1}{1} = \frac{1}{2} \not \in \mathcal{A}$$
Thus, addition is not a binary operation over $\mathcal{A}$.
\item[(e)]
Consider $\frac{a}{b}, \frac{c}{d}, \frac{e}{f} \in \mathcal{A}$. Then
$$\frac{a}{b} + \frac{c}{d} = \frac{ad + bc}{bd}$$
Note that since $b, d$ are either 1 or 2, then $bd$ equals either 1, 2, or 4. $bd = 4$ if $b = d = 2$. But then 2 divides $ad + bc$, so then $\frac{ad + bc}{bd} = \frac{a + c}{2}$, so $\frac{ad + bc}{bd} \in \mathcal{A}$. Furthermore, 
$$\left(\frac{a}{b} + \frac{c}{d} \right) + \frac{e}{f} = \frac{ad + bc}{bd} + \frac{e}{f} = \frac{adf + bcf + bde}{bdf}$$
$$ = \frac{a}{b} + \frac{cf + bd}{df} = \frac{a}{b} + \left(\frac{c}{d} + \frac{b}{f}\right)$$
Therefore, addition is associative over $\mathcal{A}$. We also have $\frac{0}{1} + \frac{a}{b} = \frac{a}{b} = \frac{0}{1}$, so $\frac{0}{1}$ is an identity under $\mathcal{A}$. And, $\frac{a}{b} + \frac{-a}{b} = 0 = \frac{-a}{b} + \frac{a}{b}$, so $\frac{-a}{b}$ is an inverse of $\frac{a}{b}$. Thus $\mathcal{A}$ is a group under addition.
\item[(f)]
$\mathcal{A}$ is not a group under addition. Consider $a = \frac{1}{3}$ and $b = \frac{1}{2}$. Then
$$a + b = \frac{1}{3} + \frac{1}{2} = \frac{5}{6} \not \in \mathcal{A}$$
Thus, addition is not a binary operation over $\mathcal{A}$.
\end{itemize}
\item[(7)]
Consider $x, y, z \in G$. Note that $x + y = [x + y] + r$, where $0 \leq r < 1$. Then $x * y = x + y - [x + y] = r$. Since $r \in G$, then $*$ is well defined on $G$. To show that $*$ is associative, we must first note that if $n$ is an integer, then $[x + n] = [x] + n$. Then,
$$(x * y) * z = (x + y - [x + y]) + z - [(x + y - [x + y]) + z]$$
$$= x + y + z - [x + y] - [x + y + z] + [x + y]$$
$$= x + y + z - [x + y + z]$$
$$= x + y + z - [y + z] + [y + z] - [x + y + z]$$
$$= x + (y + z - [y + z]) + [x + (y + z - [y + z])] = x * (y * z)$$
So, $*$ is associative over $G$. Since for any $x \in G$, $x * 0 = x + 0 - [x + 0] = x - [x] = x = 0 + x - [0 + x] = 0 * x$, then $0$ is an identity of $G$. And, for any $x \in G$, let $y = 1 - x$. Then $x * y = x + y - [x + y] = 1 - [1] = 0 = y + x - [y + x] = y * x$, so $y$ is an inverse of $x$. Therefore, $G$ is a group under $*$. To show that $G$ is abelian, we must show that $*$ is commutative. For any $x, y \in G$,
$$x * y = x + y - [x + y] = y + x - [y + x] = y * x$$
Therefore, $G$ is abelian.
\item[(8)]
\begin{itemize}
\item[(a)] Consider $z_1, z_2, z_3 \in G$. So for $n_1, n_2 \in \mathbb{Z}^+$, $z_1^{n_1} = 1$ and $z_2^{n_2} = 1$. So
$$(z_1z_2)^{n_1n_2} = z_1^{n_1n_2}z_2^{n_1n_2} = (z_1^{n_1})^{n_2}(z_2^{n_2})^{n_1} = 1^{n_2}1^{n_1} = 1$$
Therefore, multiplication is well defined over $G$.
$$(z_1z_2)z_3 = z_1z_2z_3 = z_1(z_2z_3)$$
Therefore, multiplication is associative (as a consequence of associativity of multiplying complex numbers). Since $1 \in G$, and for all $z_1$ $1z_1 = z_1 = z_11$, then 1 is an identity of $G$. Let $a = z^{n - 1}$, where $n$ is a positive integer such that $z^n = 1$. Note that
$$a^n = (z^{n-1})^n = z^{n(n - 1)} = (z^n)^{n - 1} = 1$$
Therefore, $a \in G$. Then
$$az = z^{n-1}z = z^n = 1 = zz^{n-1} = za$$
So, each $z \in G$ has an inverse in $G$. Therefore $G$ is a group under multiplication.
\item[(b)] Consider $1 \in G$. But $z = 1 + 1 = 2 \not \in G$, since $z^n$ monotonically increases as $n$ increases if $|z| > 1$. So, there is no positive. integer solution to $2^n = 1$. Therefore, addition is not well defined over $G$, and $G$ is not a group under addition.
\end{itemize}
\item[(9)]
\begin{itemize}
\item[(a)]
Let $x = x_1 + x_2\sqrt{2}, y = y_1 + y_2\sqrt{2}, z = z_1 + z_2\sqrt{2} \in G$
$$x + y = x_1 + x_2\sqrt{2} + y_1 + y_2\sqrt{2} = x_1 + y_1 + (x_2 + y_2)\sqrt{2}$$
So addition is well defined over $G$. Furthermore,
$$(x + y) + z = x_1 + y_1 + (x_2 + y_2)\sqrt{2} + z_1 + z_2\sqrt{2}$$
$$ = x_1 + x_2\sqrt{2} + y_1 + z_1 + (y_2 + z_2)\sqrt{2} = x + (y + z)$$
So addition is associative over $G$. $x + 0 = x_1 + x_2\sqrt{2} + 0 + 0\sqrt{2} = x = 0 + x$. so $0$ is an identity on $G$. Let $x^{-1} = -x_1 - x_2\sqrt{2}$ for all $x \in G$. Then $x + x^{-1} = x_1 - x_1 + x_2\sqrt{2} - x_2\sqrt{2} = 0 = x^{-1} + x$, so each $x$ has an inverse $x^{-1} = -x$. Therefore, $G$ is a group under addition.
\item[(b)]
(Note: throughout this proof, notation is abused; where $G$ is used, we mean $G - \left\lbrace 0 \right\rbrace$). Let $x = x_1 + x_2\sqrt{2}, y = y_1 + y_2\sqrt{2}, z = z_1 + z_2\sqrt{2} \in G$
$$xy = (x_1 + x_2\sqrt{2})(y_1 + y_2\sqrt{2}) = x_1y_1 + 2x_2y_2 + (x_1y_2 + x_2y_1)\sqrt{2}$$
Since $x_1y_1 + 2x_2y_2, x_1y_2 + x_2y_1 \in \mathbb{Q}$, multiplication is well defined over $G$. Furthermore,
$$(xy)z = (x_1y_1 + 2x_2y_2 + (x_1y_2 + x_2y_1)\sqrt{2})(z_1 + z_2\sqrt{2})$$
$$= x_1y_1z_1 + x_1y_1z_2\sqrt{2} + 2x_2y_2z_1 + 2x_2y_2z_2\sqrt{2}$$
$$+ x_1y_2z_1\sqrt{2} + x_2y_1z_1\sqrt{2} + 2x_1y_2z_2 + 2x_2y_1z_2$$
$$= (x_1 + x_2\sqrt{2})(y_1z_1 + 2y_2z_2 + (y_1z_2 + y_2z_1)\sqrt{2})$$
$$= x(yz)$$
So multiplication is well defined over $G$. $x1 = (x_1 + x_2\sqrt{2})(1 + 0\sqrt{2}) = x = 1x$, so 1 is an identity on $G$. Let 
$$x^{-1} = \frac{x_1}{x_1^2 - 2x_2^2} - \frac{x_2\sqrt{2}}{x_1^2 - 2x_2^2}$$
Clearly, $x^{-1} \in G$. Then
$$xx^{-1} = (x_1 + x_2\sqrt{2})\left(\frac{x_1}{x_1^2 - 2x_2^2} - \frac{x_2\sqrt{2}}{x_1^2 - 2x_2^2} \right)$$
$$= \frac{x_1^2}{x_1^2 - 2x_2^2} + \frac{x_1x_2\sqrt{2}}{x_1^2 - 2x_2^2} - \frac{x_1x_2\sqrt{2}}{x_1^2 - 2x_2^2} - \frac{2x_2^2}{x_1^2 - 2x_2^2} = 1 = x^{-1}x$$
Therefore, for all $x \in G$, $x$ has a multiplicative inverse $x^{-1} \in G$. So $G$ is a group under multiplication. 
\end{itemize}
\item[(10)]
Suppose we have a finite group $G = \left\lbrace g_1, g_2, ..., g_n\right\rbrace$ of size $n$ that has an associated group table $T$ of size $n \times n$, where an entry $t_{ij}$ if $T$ is $g_ig_j$. Recall that a $n \times n$ matrix $M$ with entries written as $m_{ij}$ is symmetric if and only if for every $i,j \in \left\lbrace 1, 2, ..., n \right\rbrace$, $m_{ij} = m_{ji}$. \\
If $G$ is abelian, then for every $i, j \in \left\lbrace 1, 2, ..., n \right\rbrace, g_ig_j = g_jg_i$, and therefore $T$ is symmetric. \\
If $T$ is symmetric, then for every $i, j \in \left\lbrace 1, 2, ..., n \right\rbrace, t_{ij} = t_{ji} \rightarrow g_ig_j = g_jg_i$, so $G$ is therefore abelian.
\item[(11)]
$\overline{0}$: $0 + 0 \equiv 0 \mod 12$, so the order of $\overline{0}$ is 1. \\
$\overline{1}$: $1 \cdot 12 \equiv 0 \mod 12$, so the order of $\overline{1}$ is 12. \\
$\overline{2}$: $2 \cdot 6 \equiv 0 \mod 12$, s the order of $\overline{2}$ is 6. \\
$\overline{3}$: $3 \cdot 4 \equiv 0 \mod 12$, s the order of $\overline{3}$ is 4. \\
$\overline{4}$: $4 \cdot 3 \equiv 0 \mod 12$, s the order of $\overline{4}$ is 3. \\
$\overline{5}$: $5 \cdot 12 \equiv 0 \mod 12$, s the order of $\overline{5}$ is 60. \\
$\overline{6}$: $6 \cdot 2 \equiv 0 \mod 12$, s the order of $\overline{6}$ is 2. \\
$\overline{7}$: $7 \cdot 12 \equiv 0 \mod 12$, so the order of $\overline{7}$ is 12. \\
$\overline{8}$: $8 \cdot 3 \equiv 0 \mod 12$, so the order of $\overline{8}$ is 2. \\
$\overline{9}$: $9 \cdot 4 \equiv 0 \mod 12$, s the order of $\overline{9}$ is 4. \\
$\overline{10}$: $10 \cdot 6 \equiv 0 \mod 12$, s the order of $\overline{10}$ is 6. \\
$\overline{11}$: $11 \cdot 12 \equiv 0 \mod 12$, s the order of $\overline{11}$ is 12.
\item[(12)]
$\overline{1}$: $1^1 \equiv 1 \mod 12$, so the order of $\overline{1}$ is 1.\\
$\overline{-1} = \overline{11}$: $11^1 \equiv 11 \mod 12, 11^2 \equiv 1 \mod 12$, so the order of $\overline{-1}$ is 2. \\
$\overline{5}$: $5^1 \equiv 5 \mod 12, 5^2 \equiv 1 \mod 12$, so the order of $\overline{5}$ is 2.\\
$\overline{7}$: $7^1 \equiv 7 \mod 12, 7^2 \equiv 1 \mod 12$, so the order of $\overline{7}$ is 2.\\
$\overline{-7} = \overline{5}$: The order of $\overline{-7}$ is the order of $\overline{5}$, which is 2.\\
$\overline{13} = \overline{1}$: The order of $\overline{13}$ is the order of $\overline{1}$, which is 1. 
\item[(13)]
$\overline{1}$: $1 \cdot 36 \equiv 0 \mod 36$, so the order of $\overline{1}$ is 36. \\
$\overline{2}$: $2 \cdot 18 \equiv 0 \mod 36$, so the order of $\overline{2}$ is 18.\\
$\overline{6}$: $6 \cdot 6 \equiv 0 \mod 36$, so the order of $\overline{6}$ is 6.\\
$\overline{9}$: $9 \cdot 4 \equiv 0 \mod 36$, so the order of $\overline{9}$ is 4.\\
$\overline{10}$: $10 \cdot 18 \equiv 0 \mod 36$, so the order of $\overline{10}$ is 18. \\ 
$\overline{12}$: $12 \cdot 3 \equiv 0 \mod 36$, so the order of $\overline{12}$ is 3.\\
$\overline{-1}$: $-1 \cdot 36 \equiv 0 \mod 36$, so the order of $\overline{-1}$ is 36.\\
$\overline{-10}$: $-10 \cdot 18 \equiv 0 \mod 36$, so the order of $\overline{-10}$ is 18.\\
$\overline{-18}$: $-18 \cdot 2 \equiv 0 \mod 36$, so the order of $\overline{-18}$ is 2.
\item[(14)]
$\overline{1}$: $1^1 \equiv 1 \mod 36$, so the order of $\overline{1}$ is 1. \\
$\overline{5}$: $5^6 \equiv 1 \mod 36$, so the order of $\overline{5}$ is 6. \\
$\overline{-5}$: $(-5)^6 \equiv 1 \mod 36$, so the order of $\overline{5}$ is 6. \\
$\overline{13}$: $13^3 \equiv 1 \mod 36$, so the order of $\overline{13}$ is 3. \\
$\overline{-13}$: $(-13)^6 \equiv 1 \mod 36$, so the order of $\overline{-13}$ is 6. \\
$\overline{17}$: $(17)^2 \equiv 1 \mod 36$, so the order of $\overline{17}$ is 2.
\item[(15)]
By Proposition 1, if $G$ is a group under operation *, then for all $a, b \in G$, $(a * b)^{-1} = b^{-1}a^{-1}$. For $a_1, a_2, ..., a_n \in G$, suppose it is true that for all $k < n$,
$$(a_1a_2...a_{k-1}a_k)^{-1} = a_k^{-1}a_{k-1}^{-1}...a_2a_1$$
Then,
$$(a_1a_2...a_n)^{-1} = ((a_1a_2...a_{n-1})a_n)^{-1}$$ 
$$= a_n^{-1}(a_1a_2...a_{n-1})^{-1} = a_n^{-1}a_{n-1}^{-1}...a_2^{-1}a_1^{-1}$$
\item[(16)]
Suppose $x^2 = 1$. Then $0 < |x| \leq 2$, so $|x|$ can only be either 1 or 2. \\
Suppose $|x| = 1$. Then 
$$x = 1 \rightarrow x \cdot x = 1 \cdot 1 \rightarrow x^2 = 1$$
Suppose $|x| = 2$. Then by definition of $|x|$,
$$x^2 = 1$$
\item[(17)]
Let $x \in G$ and $|x| = n$ for some positive integer $n$. Then
$$x^n = 1$$
Multiplying both sides by $x^{-1}$:
$$x^nx^{-1} = x^{-1} \rightarrow x^{n-1}xx^{-1} = x^{-1} \rightarrow x^{n-1} = x^{-1}$$
\item[(18)]
Suppose $xy = yx$. Multiplying both sides by $y^{-1}$:
$$y^{-1}xy = y^{-1}yx = x$$
Suppose $y^{-1}xy = x$. Multiplying both sides by $x^{-1}$:
$$x^{-1}y^{-1}xy = x^{-1}x = 1$$
Suppose $x^{-1}y^{-1}xy = 1$. Multiplying both sides by $yx$:
$$yxx^{-1}y^{-1}xy = yx$$
$$y(xx^{-1})y^{-1}xy = yx$$
$$(yy^{-1})xy = yx$$
$$xy = yx$$
\item[(19)]
\begin{itemize}
\item[(a)]
If $a = 1$ and $b = 1$, then $x^{1 + 1} = x^2 = xx = x^1x^1$. \\
Suppose for some $a, b \in \mathbb{Z}^+$, that for all $i \leq a$ and $j \leq b$, that $x^{i + j} = x^ix^j$. Without loss of generality, increment $a$ by 1. Then
$$x^{a + 1 + b} = x^{a + b}x = x^ax^bx = x^axx^b = x^{a + 1}x^b$$
If $a = 1$ and $b = 1$, then $(x^1)^1 = (x)^1 = x$. \\
Suppose for some $a, b \in \mathbb{Z}^+$, that for all $i \leq a$ and $j \leq b$, that $(x^i)^j = x^{ij}$. If $a$ is incremented by 1, then
$$(x^{a + 1})^b = (x^ax)^b = (x^a)^bx^b = x^{ab}x^b = x^{ab + b} = x^{(a + 1)b}$$
If instead $b$ is incremented by 1, then
$$(x^a)^{b + 1} = (x^{a})^bx^a = x^{ab}x^a = x^{ab + a} = x^{a(b + 1)}$$
\item[(b)]
If $a = 1$, then $(x^{1})^{-1} = x^{-1}$. \\
Suppose for some $a \in \mathbb{Z}^+$, that for all $i \leq a$, that $(x^i)^{-1} = x^{-1}$. If $a$ is incremented by 1, then
$$(x^{a + 1})^{-1} = (x^ax)^{-1} = x^{-1}(x^a)^{-1} = x^{-1}x^{-a} = x^{-1 - a} = x^{-(a + 1)}$$
Similarly, we can show that $(x^{-1})^{a + 1} = x^{-(a + 1)}$:
$$(x^{-1})^{a + 1} = (x^{-1})^ax^{-1} = x^{-a}x^{-1} = x^{-(a + 1)}$$
\item[(c)]
Let us begin with $x^{a + b} = x^ax^b$. By part (a), if $a,b > 0$, then the statement holds. Without loss of generality, let $a = 0$. Then
$$x^{a + b} = x^{0 + b} = x^b = x^0x^b$$
If $a, b < 0$, then by part (b),
$$x^{a + b} = (x^{-b - a})^{-1} = (x^{-a})^{-1}(x^{-b})^{-1} = x^ax^b$$
Suppose without loss of generality that $a < 0$ and $b > 0$. If $b > |a|$, then $a + b > 0$, and
$$x^{a + b}x^{-a}x^{-b} = x^{a + b - a}x^{-b} = x^{b}x^{-b} = 1$$
So $x^{-a}x^{-b}$ is the inverse of $x^{a + b}$. But $x^{-a}x^{-b}$ is also the inverse of $x^ax^b$. Since inverses are unique, $x^{a + b} = x^ax^b$. \\
If $b = |a|$, then $a + b = 0$, and
$$x^{a + b} = x^{0} = 1 = x^{-b}x^b = x^ax^b$$
If $b < |a|$, then $a + b < 0$, and we have already shown that $x^{a + b} = x^ax^b$. \\
Let us now consider $(x^a)^b = x^{ab}$. By part (a), if $a, b > 0$, then the statement holds. If $a = 0$, then
$$(x^0)^b = 1^b = 1 = x^0 = x^{0b}$$
If $b = 0$, then
$$(x^a)^b = (x^a)^0 = 1 = x^0 = x^{a0}$$
If $a > 0$ and $b < 0$, then
$$(x^a)^b = ((x^a)^{-b})^{-1} = (x^{-ab})^{-1} = x^{ab}$$
If $b > 0$ and $a < 0$, then 
$$(x^a)^b = ((x^{-a})^{-1})^b = ((x^{-a})^b)^{-1} = (x^{-ab})^{-1} = x^{ab}$$
If $a, b < 0$, then
$$(x^a)^b = ((((x^{-a})^{-1})^{-b})^{-1} = (((x^{-a})^{-b})^{-1})^{-1} = x^{ab}$$
\end{itemize}
\item[(20)]
Suppose $x$ has infinite order. Suppose $x^{-1}$ has finite order $m$. That is,
$$(x^{-1})^m = 1$$
Then
$$1 = (x^{-1})^m(x^{-1})^{-m} = (x^{-1})^{-m} = x^m$$
But this implies $x$ has a finite order $\leq m$; a contradiction. Therefore, $x^{-1}$ has infinite order. By similar reasoning, if $x^{-1}$ has infinite order, then $x$ must also have infinite order. \\
Suppose $n$ is the order of $x$ and $m$ is the order of $x^{-1}$. Then
$$1 = x^n(x^{-1})^m = x^nx^{-m} = x^{n - m}$$
If $n \neq m$, then $0 < n - m < n$. But this contradicts that $n$ is the order of $x$. Therefore, $n = m$.
\item[(21)]
If $x \in G$ has finite order $n$, where $n = 2k + 1$ for some $k \geq 0 \in \mathbb{Z}$ is odd, then
$$x = x^nx = x^{n + 1} = x^{2k + 2} = x^{2(k + 1)} = (x^2)^{k + 1}$$
\item[(22)]
Let $x, g \in G$ with finite order, and $x$ has finite order $|x|$. Then
$$x^{|x|} = 1 \rightarrow (g^{-1})^{|x|}x^{|x|} = (g^{-1})^{|x|}$$
$$\rightarrow (g^{-1})^{|x|}x^{|x|}g^{|x|} = (g^{-1})^{|x|}g^{|x|} = (g^{-1}g)^{|x|} = 1^{|x|} = 1$$
$$\rightarrow (g^{-1}xg)^{|x|} = 1$$
Therefore, $|x| = |g^{-1}xg|$. \\
Let $x = ab$, and $g = a$ for $a, b \in G$. Then
$$|ab| = |(a)^{-1}ab(a)| = |ba|$$
\item[(23)]
If $n$ is the finite order of $x$, and $n = st$, then by Exercise 19,
$$x^n = 1 \rightarrow x^{st} = 1 \rightarrow (x^s)^t = 1$$
Thus, $|x^s| = t$
\item[(24)] Let $n = 0$. Then $(ab)^0 = 1 = a^0b^0$. Suppose for some $n$, that for all $0 \leq k < n$,
$$(ab)^k = a^kb^k$$
Then
$$(ab)^n = (ab)^{n - 1}ab = a^{n-1}b^{n-1}ab$$
$$= a^{n-1}b^{n-1}ba = a^{n-1}b^na = a^{n-1}ab^n = a^nb^n$$
If $n < 0$, then
$$(ab)^n = ((ab)^{-n})^{-1} = (a^{-n}b^{-n})^{-1} = b^na^n = a^nb^n$$
Note that $a$ and $b$ commute throughout this proof.
\item[(25)]
Suppose $x^2 = 1$ for all $x \in G$. Then $x = x^{-1}$ So for $a, b \in G$, we have
$$ab = (ab)^{-1} = b^{-1}a^{-1} = ba$$
\item[(26)]
From the problem statement, we know that * is well defined on $H$. Let $a, b, c \in H$. Since $H \subseteq G$, then $a, b, c \in G$. Therefore, $(a * b) * c = a * (b * c)$. From the problem statement, we also know that for each $a \in H$, $a^{-1} \in H$. So $aa^{-1} = e \in H$ is the identity of $H$. Thus $H$ is a group under *.
\item[(27)]
Let $\mathcal{A} = \left\lbrace x^n | n \in \mathbb{Z}\right\rbrace$ for $x \in G$. Note for each $a \in \mathcal{A}$, that $a \in G$ since * is well defined over $G$. Consider $a_1, a_2, a_3 \in \mathcal{A}$, where for $i = 1, 2, 3$, $a_i = x^{n_i}$ for $n_i \in \mathbb{Z}$. Since $n_1 + n_2 \in \mathbb{Z}$, then
$$a_1 * a_2 = x^{n_1}x^{n_2} = x^{n_1 + n_2} \in \mathcal{A}$$
So * is well defined over $\mathcal{A}$. Furthermore,
$$(a_1 * a_2) * a_3 = (x^{n_1}x^{n_2})x^{n_3} = x^{n_1 + n_2}x^{n_3} = x^{n_1 + n_2 + n_3}$$ 
$$= x^{n_1}(x^{n_2 + n_3}) = x^{n_1}(x^{n_2}x^{n_3}) = a_1 * (a_2 * a_3)$$
Thus, * is an associative operator. Since 1 is the identity in $G$, and $1 = x^0 \in \mathcal{A}$, then $\mathcal{A}$ has the identity 1. Let $a_1^{-1} = x^{-n_1} \in \mathcal{A}$. Then
$$a_1a_1^{-1} = x^{n_1}x^{-n_1} = x^{n_1 - n_1} = x^0 = 1 = a_1^{-1}a_1$$
Thus, $a_1^{-1}$ is the inverse of $a_1$, so every $a \in \mathcal{A}$ has an inverse. Therefore $\mathcal{A}$ is a subgroup of $G$.
\item[(28)]
\begin{itemize}
\item[(a)]
$$(a_1, b_1)[(a_2, b_2)(a_3, b_3)] = (a_1, b_1)(a_2 * a_3, b_2 \diamond b_3)$$
$$= (a_1 * (a_2 * a_3), b_1 \diamond (b_2 \diamond b_3)) = ((a_1 * a_2) * a_3, (b_1 \diamond b_2) \diamond b_3)$$
$$= (a_1 * a_2, b_1 \diamond b_2)(a_3, b_3) = [(a_1, b_1)(a_2, b_2)](a_3, b_3)$$
\item[(b)] Let $a \in A$, $b \in B$. Then
$$(1, 1)(a, b) = (1 * a, 1 \diamond b) = (a, b) = (a, b)(1, 1)$$
\item[(c)]
$$(a, b)(a^{-1}, b^{-1}) = (a * a^{-1}, b \diamond b^{-1}) = (1, 1) = (a^{-1}, b^{-1})(a, b)$$
\end{itemize}
\item[(29)]
Suppose $A \times B$ is abelian. Then for $(a_1, b_1), (a_2, b_2) \in A \times B$,
$$(a_1 * a_2, b_1 \diamond b_2) = (a_1, b_1)(a_2, b_2) =  (a_2, b_2)(a_1, b_1) = (a_2 * a_1, b_2 \diamond b_1)$$
So, $a_1 * a_2 = a_2 * a_1$ and $b_1 \diamond b_2 = b_2 \diamond b_1$, so $A$ and $B$ are abelian. \\
Suppose $A$ and $B$ are abelian. Then for $a_1, a_2 \in A$ and $b_1, b_2 \in B$, then
$$(a_1, b_1)(a_2, b_2) = (a_1 * a_2, b_1 \diamond b_2) = (a_2 * a_1, b_2 \diamond b_1) =  (a_2, b_2)(a_1, b_1)$$
Therefore, $A \times B$ is abelian.
\item[(30)]
$$(a, 1)(1, b) = (a * 1, 1 \diamond b) = (1 * a, b \diamond 1) = (1, b)(a, 1)$$
Let $n$ be the order of $(a, b)$. Then for some $i, j \in \mathbb{Z}^+$
$$(a, b)^n = (a^n, b^n) = (a^{i|a|}, b^{j|b|}) = (1, 1)$$
So,
$$n = i|a| = j|b|$$
$|a| | n$ and $|b| | n$, so $n$ is a multiple of $|a|$ and $|b|$. Furthermore, suppose there exists some positive integer $m$ such that
$$(a, b)^m = (1, 1)$$
and also, that $|a| | m$, $|b| | m$, but $n \not | m$. That is, $l$ is not the least common multiple. But then $m < n$, so $n$ cannot be the order of $(a, b)$. So by contradiction, $l$ must be the least common multiple.
\item[(31)]
Let $G$ be a group of even order. Define $t(G)$ to be
$$t(G) = \left\lbrace g \in G | g \neq g^{-1} \right\rbrace$$
Let
$$A = \left\lbrace \left\lbrace a, a^{-1} \right\rbrace | a \in t(G) \right\rbrace$$
Since $\bigcup_{\alpha \in A}\alpha = t(G)$, and for $\alpha, \beta \in A$, $\alpha \cap \beta = \emptyset$, then $A$ is a partition of $t(G)$. Then
$$|t(G)| = \sum_{\alpha \in A}|\alpha| = 2|A|$$
For every $a \in G - t(G)$, $a = a^{-1}$. If $a \neq 1$, then $a^2 = aa^{-1} = 1$. So $a$ has order 2. Note that the identity 1 is always an element of $G - t(G)$, so $G - t(G)$ is always nonempty. If $|G - t(G)| = 1$, then $|G| = |t(G)| + |G - t(G)| = 2|A| + 1$. But this is a contradiction, since $|G|$ is even. Therefore $|G - t(G)| > 1$, so there must exist some element of order $2$ in $G$.
\item[(32)]
Suppose $x \in G$ has order $n$, and that for sake of contradiction that $1, x, x^2, ..., x^{n-1}$ are not all distinct. Then for some $i,j \in \mathbb{Z}^+$ where $i < j < n$, that
$$x^j = x^i \rightarrow x^{j - i} = 1$$
But $1 \leq j - i < n$, contradicting that $x$ has order $n$. Thus $1, x, x^2, ..., x^{n-1}$ must all be distinct. It then follows that since there are at least $|x|$ distinct elements of $G$ that $|x| \leq |G|$.
\item[(33)]
\begin{itemize}
\item[(a)]
Proving the contrapositive: suppose that
$$x^i = x^{-i}$$
For $1 \leq i \leq n - 1$. Then
$$x^n = 1 \rightarrow x^nx^{-i} = x^{-i} \rightarrow x^nx^{-i} = x^i \rightarrow x^n = x^{2i}$$
Ie. $n$ is even. 
\item[(b)]
Suppose $n = 2k$ and $x^i = x^{-i}$. Then
$$x^i = x^{-i} \rightarrow x^{2i} = 1 \rightarrow x^{2i} = x^{2k} \rightarrow i = k$$
Suppose $i = k$. Then
$$x^{2k} = 1\rightarrow x^{2i} = 1 \rightarrow x^i = x^{-i}$$
\end{itemize}
\item[(34)]
Suppose $x$ has infinite order and suppose for sake of contradiction that $x^n, n \in \mathbb{Z}$ are not all distinct. Then for some $i, j \in \mathbb{Z}$ where $j > i$, 
$$x^j = x^i \rightarrow x^{j - i} = 1$$
But this implies $x$ has finite order at most $j - i$. Thus by contradiction, all $x^n$ must be disinct.
\item[(35)]
Consider $x^a$ an integral power of $x$. By the Division Algorithm, for $q, r \in \mathbb{Z}$, $0 \leq r < n$,
$$a = qn + r$$
Then
$$x^a = x^{qn + r} = x^{qn}x^r = (x^n)^qx^r = x^r$$
Note that $x^r \in \left\lbrace 1, x, x^2, ..., x^{n - 1} \right\rbrace = \mathcal{A}$, so therefore $x^a \in \mathcal{A}$.
\item[(36)]
Consider the expression $ab$. If $ab = b$, then $a = 1$, so then $ab \neq b$. Similarly, $ab \neq a$. Suppose $ab = 1$. Since $b \neq a$, then $a^2 \neq ab = 1$. Since the order of $a$ is at most 3, then $a^3 = 1$, and $a^3b = b \rightarrow a^2 = b$. Similarly, $b^2 = a$. Consider $ac$. If $ac = 1$, then by similar reasoning $a^2 = c$ and $c^2 = a$. But then $b = c$, a contradiction. Furthermore, using similar reasoning as above $ac \neq a$ and $ac \neq c$. So therefore $ac = b$. But then $a^3c = a^2b \rightarrow c = a$, a contradiction. So therefore the initial assumption of $ab = 1$ is wrong. Therefore $ab = c$. Similarly, $ba = c$, $ac = ca = b$, and $bc = cb = a$. Then $a^2b = ac = b \rightarrow a^2 = 1$. Similarly, $b^2 = 1$ and $c^2 = 1$. The final group table is then \\
\begin{center}
\begin{tabular}{| c || c | c | c | c |}
\hline
* & 1 & $a$ & $b$ & $c$ \\
\hline
\hline
1 & 1 & $a$ & $b$ & $c$ \\
\hline
$a$ & $a$ & 1 & $c$ & $b$ \\
\hline
$b$ & $b$ & $c$ & 1 & $a$ \\
\hline
$c$ & $c$ & $b$ & $a$ & 1 \\
\hline
\end{tabular}
\end{center}
By inspection, the group table is symmetric, so $G$ is abelian.
\end{itemize}
\end{document}