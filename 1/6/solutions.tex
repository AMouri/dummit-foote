\documentclass[12pt]{article}
\usepackage{amsmath, amssymb}
\begin{document}
\title{Introduction to Groups - Homomorphisms and Isomorphisms}
\author{Alec Mouri}

\maketitle
\section*{Exercises}
\begin{itemize}
\item[(1)]
\begin{itemize}
\item[(a)]
Suppose $n = 1$. Clearly, $\varphi(x) = \varphi(x)$. Suppose for some $n \in \mathbb{Z}^+$ that $\varphi(x^n) = \varphi(x)^n$. Then:
$$\varphi(x^{n+1}) = \varphi(x^nx) = \varphi(x^n)\varphi(x) = \varphi(x)^n\varphi(x) = \varphi(x)^{n+1}$$
\item[(b)]
Lemma: $\varphi(1) = 1$. Proof: for some $x \in G$:
$$\varphi(x) = \varphi(x)\varphi(1) \rightarrow 1 = \varphi(1)$$
Therefore, by the Lemma,
$$1 = \varphi(1) = \varphi(xx^{-1}) = \varphi(x)\varphi(x^{-1}) \rightarrow \varphi(x^{-1}) = \varphi(x)^{-1}$$
Let $n = 0$. Then $\varphi(x^0) = \varphi(1) = 1 = \varphi(x)^0$

Let $n < 0$. Then
$$\varphi(x^n) = \varphi((x^{-n})^{-1}) = \varphi(x^{-n})^{-1} = (\varphi(x)^{-n})^{-1} = \varphi(x)^n$$
\end{itemize}
\item[(2)]
Suppose $\varphi$ is an isomorphism. Suppose $|\varphi(x)| = \infty$, and for sake of contradiction that $|x| = n < \infty$. Then $\varphi(x)^n = \varphi(x^n) = \varphi(1) = 1 \rightarrow |\varphi(x)| \leq n$, a contradiction. Thus, $|x| = \infty$. Suppose $|x| = \infty$, and for sake of contradiction that $|\varphi(x)| = n < \infty$. Then $\varphi(1) = 1 = \varphi(x)^n = \varphi(x^n) \rightarrow 1 = x^n \rightarrow |x| \leq n$, a contradiction. Thus $|\varphi(x)| = \infty$. Suppose $|x|$ and $|\varphi(x)|$ are finite. Then we have
$$1 = \varphi(1) = \varphi(x^{|x|}) = \varphi(x)^{|x|}$$
Suppose for some $1 \leq i < |x|$, that $\varphi(x)^i = 1$. Then 
$$1 = \varphi(x)^i = \varphi(x^i) \rightarrow x^i = 1$$
But, this implies that $|x|$ is not the order of $x$, a contradiction. Therefore, no such $i$ exists such that $\varphi(x)^i = 1$, and therefore $|\varphi(x)| = |x|$.

Let $G_n$ be the subset of $G$ containing all elements of $G$ that have order $n$, and let $H_n$ be the subset of $H$ containing all elements of $H$ that have order $n$. Since $\varphi$ is a bijection, and for all $x \in G_n$, $\varphi(x) \in H_n$, then $|G_n| = |H_n|$.

The above result does not hold if $\varphi$ is a homomorphism, but not an isomorphism. Consider the Quaternion group $Q_8$. Consider the cyclic subgroup $q = \left\lbrace i^n | n \in \mathbb{Z} \right\rbrace = \left\lbrace 1, -1, i, -i \right\rbrace$ generated by $i$. Define $\varphi: q \rightarrow Q_8$ such that for all $x \in q$: $\varphi(x) = x$. Clearly, $\varphi$ is a homomorphism, and not an isomorphism. $q$ contains 2 elements of order 4, $i$ and $-i$. But, $Q_8$ contains 6 elements of order 4: $i, -i, j, -j, k, -k$.
\item[(3)]
Suppose $G$ is abelian. Since $\varphi$ is surjective, then for $h_1, h_2 \in H$, there exists some $g_1, g_2 \in G$, $\varphi(g_1) = h_1$ and $\varphi(g_2) = h_2$. Then
$$h_1h_2 = \varphi(g_1)\varphi(g_2) = \varphi(g_1g_2) = \varphi(g_2g_1) = \varphi(g_2)\varphi(g_1) = h_2h_1$$
Thus $H$ is abelian. Now suppose $H$ is abelian. Since $\varphi$ is injective, then for some $g_1, g_2 \in G$ where $g_1 \neq g_2$, then $\varphi(g_1) \neq \varphi(g_2)$. Therefore,
$$\varphi(g_1g_2) = \varphi(g_1)\varphi(g_2) = \varphi(g_2)\varphi(g_1) = \varphi(g_2g_1)$$
Since $\varphi$ is injective, then $g_1g_2 = g_2g_1$, and $G$ is abelian.

If $\varphi$ is a homomorphism, then $\varphi$ must be surjective for $H$ to be abelian if $G$ is abelian. See the cyclic subgroup of the quaternion group in the solution for exercise 2: $\varphi$ from $q$ to $Q_8$ is not surjective, and $q$ is abelian, but $Q_8$ is not abelian.
\item[(4)]
Suppose $\mathbb{R} - \left\lbrace 0 \right\rbrace$ and $\mathbb{C} - \left\lbrace 0 \right\rbrace$ are isomorphic. From Exercise 2, for each $n \geq 1$, two isomorphic groups have the same number of elements with order $n$. Note that for $x \in \mathbb{R} - \left\lbrace 0 \right\rbrace$, if $x = 1$, then $|x| = 1$, and if $x = -1$, then $|x| = 2$. If $|x| > 1$ (where the bars denote the magnitude of $x$), then $|x|^m > 1$ for all $m \geq 1$, so the order of $x$ is infinite. Similarly, if $|x| < 1$, then $|x|^m < 1$ for all $m \geq $, so the order of $x$ is again infinite. However, consider $i \in \mathbb{C} - \left\lbrace 0 \right\rbrace$. $i$ has order 4, but $\mathbb{R} - \left\lbrace 0 \right\rbrace$ has no elements of order 4. Therefore, the groups are not isomorphic.
\item[(5)]
Since $\mathbb{Q}$ is countable and $\mathbb{R}$ is not, then there does not exist a bijection between $\mathbb{Q}$ and $\mathbb{R}$, so therefore they are not isomorphic.
\item[(6)]
Lemma: Suppose $\varphi: G \rightarrow H$ is a surjective homomorphism. If $G$ is generated by one element, then $H$ must also be generated by one element. Proof: Let $g$ be the generator of $G$. For any $a \in G$, then $a = g^k$ for some $k \in \mathbb{Z}$. For any $b \in H$, then for some $c \in G$, $\varphi(c) = b$. Therefore,
$$b = \varphi(c) = \varphi(g^k) = \varphi(g)^k$$
That is, $\varphi(g)$ generates $H$.

Suppose $\varphi$ is a bijection between $\mathbb{Z}$ and $\mathbb{Q}$, and that they are isomorphic. Note that 1 generates $\mathbb{Z}$, so therefore by the Lemma some $x \in \mathbb{Q}$ must be a generator. Suppose $\frac{i}{j}$ generates $\mathbb{Q}$. Note that since $\frac{1}{j} \in \mathbb{Q}$, and for all $k \in \mathbb{Z}$, $\frac{ki}{j} \neq \frac{1}{j}$ unless $i = 1$. Consider $\frac{1}{a} \in \mathbb{Q}$, where $a > j$. But for any $k \in \mathbb{Z}, \frac{k}{j} \neq \frac{1}{a}$. Therefore, $\frac{1}{j}$ cannot generate $\mathbb{Q}$. Then $\mathbb{Z}$ and $\mathbb{Q}$ cannot be isomorphic.
\item[(7)]
There are 2 elements of $D_8$ that have order 4: $r$ and $r^3$. But there are 6 elements of $Q_8$ that have order 4: $i, -i, j, -j, k$, and $-k$. Therefore by Exercise 2, $D_8$ and $Q_8$ are not isomorphic.
\item[(8)]
Note that $|S_n| = n!$, and $|S_m| = m!$. Thus if $n \neq m$, then $|S_n| \neq |S_m|$, so therefore $S_n$ and $S_m$ are not isomorphic.
\item[(9)]
There is at least 1 element of $D_{24}$ that has order 12: $r$. But, there are no elements of $S_4$ that have order 12. In particular, the maximum order of any element of $S_4$ is 4. Thus, $D_{24}$ and $S_4$ are not isomorphic.
\item[(10)]
\begin{itemize}
\item[(a)]
Choose $a \in \Delta$ and $b \in \Omega$ such that $\theta(a) = b$. Then
$$\varphi(\sigma)(b) = \theta(\sigma(\theta^{-1}(b))) = \theta(\sigma(a)) \in \Omega$$
So, $\varphi$ is well defined
\item[(b)]
Define $\eta: S_\Omega \rightarrow S_\Delta$ to be $\eta(\tau) = \theta^{-1} \circ \tau \circ \theta$. Then
$$\eta(\varphi(\sigma)) = \theta^{-1} \circ \theta \circ \sigma \circ \theta^{-1} \circ \theta = \sigma$$
And
$$\varphi(\eta(\tau)) = \theta \circ \theta^{-1} \circ \tau \circ \theta \circ theta^{-1} = \tau$$
Since $\eta$ is a two-sided inverse of $\varphi$, then $\varphi$ is a bijection.
\item[(c)]
$$\varphi(\sigma \circ \tau) = \theta \circ \sigma \circ \tau \circ \theta^{-1} = \theta \circ \sigma \circ \theta^{-1} \circ \theta \circ \tau \circ \theta^{-1} = \varphi(\sigma) \circ \varphi(\tau)$$
\end{itemize}
\item[(11)]
Define $\varphi: A \times B \rightarrow B \times A$ to be $\varphi((a, b)) = (b, a)$. Consider $(a, b), (c, d) \in A \times B$. Then
$$\varphi((a, b)(c, d)) = \varphi((ac, bd)) = (bd, ac) = (b, a)(d, c) = \varphi(a, b)\varphi(d, c)$$
So, $\varphi$ is homomorphic. Moreover, $\varphi$ is a bijection, define $\eta: B \times A \rightarrow A \times B$ to be $\eta((b, a)) = (a, b)$. Then $\varphi(\eta(b, a)) = (b, a)$, and $\eta(\varphi(a, b)) = (a, b)$ for $a \in A$ and $b \in B$. Therefore, $\varphi$ is an isomorphism, and $A \times B \cong B \times A$. 
\item[(12)]
Define $\varphi: (A \times B) \times C \rightarrow A \times (B \times C)$ to be $\varphi(((a, b), c)) = (a, (b, c))$. Consider $((a, b), c), ((d, e), f) \in (A \times B) \times C$. Then
$$\varphi(((a, b), c)((d, e), f)) = \varphi(((ad, be), cf)) = (ad, (be, cf))$$
$$= (a, (b, c))(d, (e, f)) = \varphi(((a, b), c))\varphi(((d, e), f))$$
So, $\varphi$ is homomorphic. Moreover, $\varphi$ is a bijection, define $\eta: A \times (B \times C) \rightarrow (A \times B) \times C$ to be $\eta((a, (b, c))) = ((a, b), c)$. Then $\varphi(\eta(a, (b, c))) = (a, (b, c))$, and $\eta(\varphi(((a, b), c))) = ((a, b), c)$ for $a \in A$, $b \in B$, $c \in C$. Therefore, $\varphi$ is an isomorphism, and $(A \times B) \times C \cong A \times (B \times C)$.
\item[(13)]
Let $\mathcal{H}$ be the image of $\varphi$. Clearly, $\mathcal{H} \subseteq H$. Suppose we have $h_1, h_2 \in H$ such that there exists $g_1, g_2 \in G$ such that $\varphi(g_1) = h_1, \varphi(g_2) = h_2$, ie. $h_1, h_2 \in \mathcal{H}$. Then
$$h_1h_2 = \varphi(g_1)\varphi(g_2) = \varphi(g_1g_2)$$
Ie. $h_1h_2 \in\mathcal{H}$, so therefore $\mathcal{H}$ is closed under the group operation. Furthermore, for some $g_1 \in G$ with image $h_1$, then
$$\varphi(g_1^{-1}) = \varphi(g_1)^{-1} = h_1^{-1}$$
Ie. $h_1^{-1} \in \mathcal{H}$, so therefore $\mathcal{H}$ is closed under inverses. Thus, $\mathcal{H}$ is a subgroup of $H$.

Suppose $\varphi$ is injective. Let $\tau: G \rightarrow \mathcal{H}$ to be defined as: for $g \in G$, then $\tau(g) = \varphi(g)$. Clearly, $\tau$ is bijective and homomorphic, and is therefore an isomorphism. Thus, $G \cong \mathcal{H}$.
\item[(14)]
Let $\mathcal{K}$ be the kernel of $\varphi$. Suppose $k_1, k_2 \in \mathcal{K}$. Then 
$$\varphi(k_1k_2) = \varphi(k_1)\varphi(k_2) = 1$$
Thus, $k_1, k_2 \in \mathcal{K}$, and so $\mathcal{K}$ is closed under the group operation. Furthermore, for $k \in \mathcal{K}$, then
$$\varphi(k^{-1}) = \varphi(k)^{-1} = 1$$
Ie, $k^{-1} \in \mathcal{K}$, and $\mathcal{K}$ is closed under inverses. Thus, $\mathcal{K}$ is a subgroup of $G$.

If $\varphi$ is injective, then there is precisely one element $g \in G$ such that $\varphi(g) = 1$. In particular, $g = 1$ (from the lemma from Exercise 1). Therefore, $\mathcal{K}$ is the identity subgroup.

If $\varphi$ is not injective, then for some distinct $g_1, g_2 \in G$, then $\varphi(g_1) = \varphi(g_2)$. Then
$$1 = \varphi(g_1)\varphi(g_2)^{-1} = \varphi(g_1)\varphi(g_2^{-1}) = \varphi(g_1g_2^{-1})$$
Since inverses are distinct, then $g_1g_2^{-1} \neq 1$, and $g_1g_2^{-1} \in G$. Therefore, $\mathcal{K}$ is not the identity subgroup of $G$.
\item[(15)]
Let $(x_1, y_1), (x_2, y_2) \in \mathbb{R}^2$. Then
$$\pi((x_1, y_1)(x_2, y_2)) = \pi((x_1x_2, y_1y_2)) = x_1x_2 = \pi((x_1, y_1))\pi((x_2, y_2))$$
Therefore, $\pi$ is a homomorphism.

Let $\Pi$ be the kernel of $\pi$. Note that for all $(x, y) \in \mathbb{R}^2$, $\pi((x, y)) = 1$ only if $x = 1$. Therefore,
$$\Pi = \left\lbrace (1, y) \in \mathbb{R}^2 \right\rbrace$$
\item[(16)]
Let $(a_1, b_1), (a_2, b_2) \in G$. Then
$$\pi_1((a_1, b_1)(a_2, b_2)) = \pi_1((a_1a_2, b_1b_2)) = a_1a_2 = \pi_1((a_1, b_1))\pi_1((a_2, b_2))$$
And
$$\pi_2((a_1, b_1)(a_2, b_2)) = \pi_2((a_1a_2, b_1b_2)) = b_1b_2 = \pi_2((a_1, b_1))\pi_2((a_2, b_2))$$
Thus, $\pi_1$ and $\pi_2$ are homomorphisms.

Let $\Pi_1$ be the kernel of $\pi_1$, and $\Pi_2$ be the kernel of $\pi_2$. Note that for $(a, b) \in G$, $\pi_1((a, b)) = a$ only if $a = 1$, and $\pi_2((a, b)) = b$ only if $b = 1$. Therefore,
$$\Pi_1 = \left\lbrace (1, b) \in G \right\rbrace, \Pi_2 = \left\lbrace (a, 1) \in G \right\rbrace$$
\item[(17)]
Let $\varphi \equiv g \mapsto g^{-1}$

Suppose $\varphi$ is a homomorphism. Then for $a, b \in G$,
$$ab = \varphi((ab)^{-1}) = \varphi(b^{-1}a^{-1}) = \varphi(b^{-1})\varphi(a^{-1}) = ba$$
Thus, $G$ is abelian.

Suppose $G$ is abelian. Then for $a, b \in G$,
$$\varphi(a^{-1})\varphi(b^{-1}) = ab = \varphi((ab)^{-1}) = \varphi(b^{-1}a^{-1}) = \varphi(a^{-1}b^{-1})$$
Thus $\varphi$ is a homomorphism.
\item[(18)]
Let $\varphi \equiv g \mapsto g^2$.

Suppose $\varphi$ is a homomorphism. Then for $a, b \in G$,
$$abab = (ab)^2 = \varphi(ab) = \varphi(a)\varphi(b) = a^2b^2 = aabb \rightarrow ba = ab$$
Therefore, $G$ is abelian.

Suppose $G$ is abelian. Then for $a, b \in G$,
$$\varphi(a)\varphi(b) = a^2b^2 = (ab)^2 = \varphi(ab)$$
Thus, $\varphi$ is a homomorphism.
\item[(19)]
Let $\varphi \equiv z \mapsto z^k$. Let $z_1, z_2 \in G$. Note that since $G \subseteq \mathbb{C}$, then $G$ is abelian. Then
$$\varphi(z_1z_2) = (z_1z_2)^k = z_1^kz_2^k = \varphi(z_1)\varphi(z_2)$$
Therefore, $\varphi$ is a homomorphism. Define $\eta \equiv z \mapsto z^{1/k}$. Since
$$(z^{1/k})^{nk} = z^{n} = 1$$
Then $\eta$ maps from $G$ to itself. And,
$$\varphi(\eta(z)) = \varphi(z^{1/k}) = z$$
Thus, $\eta$ is a right inverse of $\varphi$, and therefore $\varphi$ is a surjection.

Consider $z = e^{\frac{2\pi i}{k}} = \cos\left(\frac{2\pi}{k}\right) + i\sin\left(\frac{2\pi}{k}\right)$. Then $z^k = e^{2\pi i} = 1$, so $z \in G$. And, $\varphi(z) = z^k = 1$. But $z \neq 1$, so therefore $\varphi$ is not injective.
\item[(20)]
Function composition is associative. Consider the trivial isomorphism: $\varphi: g \mapsto g$ for $g \in G$. Then for any $\eta \in AUT(G)$,
$$(\eta \circ \varphi)(g) = \eta(\varphi(g)) = \eta(g)$$
And
$$(\varphi \circ \eta)(g) = \varphi(\eta(g)) = \eta(g)$$
Thus, $\varphi$ is the identity isomorphism of $AUT(G)$. And, for any $\eta \in AUT(G)$, then define $\eta^{-1}: g \rightarrow h$, where $h$ is the fiber of $\eta$ over $g$ - Note that $|h| = 1$ since $\eta$ is a bijection. Then
$$(\eta \circ \eta^{-1})(g) = \eta(h) = g$$
And
$$(\eta^{-1} \circ \eta)(g) = g$$
Thus, $\eta \circ \eta^{-1} = \eta^{-1} \circ \eta = \varphi$, so therefore $\eta$ $AUT(G)$ is closed under inverses. Thus, $AUT(G)$ is a group under function composition.
\item[(21)]
Let $\varphi \equiv q \mapsto kq$. Consider $a, b \in \mathbb{Q}$. Then
$$\varphi(a + b) = k(a + b) = ka + kb = \varphi(a) + \varphi(b)$$
So $\varphi$ is a homomorphism. Let $q_1, q_2 \in \mathbb{Q}$, where $q_1 \neq q_2$. Then $\varphi(q_1) = kq_1 \neq kq_2 = \varphi(q_2)$, so $\varphi$ is injective. And, for $q \in \mathbb{Q}$, $\varphi(\frac{q}{k}) = q$, so therefore $\varphi$ is surjective. Thus, $\varphi$ is bijective, and an isomorphism.
\item[(22)]
Let $\varphi \equiv a \mapsto a^k$. Consider $a, b \in A$. Then
$$\varphi(ab) = (ab)^k = a^kb^k = \varphi(a)\varphi(b)$$
So, $\varphi$ is a homomorphism. Suppose $k = -1$. Since inverses are unique, then $\varphi$ is injective. And, for $a \in A$, $\varphi(a^{-1}) = a$, so $\varphi$ is surjective. Therefore, $\varphi$ is bijective, and an isomorphism.
\item[(23)]
Define a map $\varphi: G \rightarrow G \equiv x \mapsto x^{-1}\sigma(x)$. I claim that $\varphi$ is bijection. Let $a, b \in G$, where $a \neq b$. Then $\varphi(a) = a^{-1}\sigma(a)$, and $\varphi(b) = b^{-1}\sigma(b)$. Suppose $\varphi(a) = \varphi(b)$. Then 
$$a^{-1}\sigma(a) = b^{-1}\sigma(b) \rightarrow ba^{-1} = \sigma(ba^{-1})$$
But this is only true if $ba^{-1} = 1$, implying that $a = b$, a contradiction . Therefore, $\varphi$ is injective. Since $G$ is finite, then $\varphi$ is also surjective, and a bijection.

Thus, for any $g \in G$, then $\exists x \in G$ such that $g = x^{-1}\sigma(x)$. Consider $g \in G$. Then for some $x \in G$,
$$\sigma(g) = \sigma(x^{-1}\sigma(x)) = \sigma(x)^{-1}x = (x^{-1}\sigma(x))^{-1} = g^{-1}$$
Therefore, for $a, b \in G$, then
$$ab = ((ab)^{-1})^{-1} = \sigma((ab)^{-1}) = \sigma(b^{-1}a^{-1}) = \sigma(b^{-1})\sigma(a^{-1}) = ba$$
Thus, $G$ is abelian.
\item[(24)]
Define $\varphi: G \rightarrow D_{2n}$ by $\varphi(x^a(xy)^b) = s^ar^b$, where $0 \leq a < 2$ and $0 \leq b < n$, ie. replace $x$ with $s$ and $xy$ with $r$. We know that $x^2 = 1$, $(xy)^n = 1$, and from Chapter 1, Section 2, Exercise 6, $(xy)x = x(xy)^{-1}$, thus the relations of $G$ are also satisfied by $D_{2n}$ under $\varphi$: thus the mapping is well defined. Consider $x^a(xy)^b, x^c(xy)^d \in G$. Then
$$\varphi(x^a(xy)^bx^c(xy)^d) = \varphi(x^{a+c}(xy)^{d - b}) = s^{a+c}r^{d - b}$$
$$= s^as^cr^{-b}r^d =  s^ar^bs^cr^d = \varphi(x^a(xy)^b)\varphi(x^c(xy)^d)$$
Thus, $\varphi$ is a homomorphism. Since $s, r$ generate $D_{2n}$, then $\varphi$ is surjective. And, since $x(xy) = y$, then $x, xy$ generate $G$, and $|G| = |D_{2n}|$, so $\varphi$ is injective and bijective. Therefore, $\varphi$ is an isomorphism, and $G \cong D_{2n}$.
\item[(25)]
\begin{itemize}
\item[(a)]
Note that 
$$\begin{pmatrix}
1 \\
0
\end{pmatrix}, \begin{pmatrix}
0 \\
1
\end{pmatrix}$$
generate the $x, y$ plane. In particular, for any
$$\begin{pmatrix}
x \\
y
\end{pmatrix} \in \mathbb{R}^2$$
Then
$$\begin{pmatrix}
x \\
y
\end{pmatrix} = x\begin{pmatrix}
1 \\
0
\end{pmatrix} + y\begin{pmatrix}
0 \\
1
\end{pmatrix}$$
If $(1, 0)^\top$ is rotated by $\theta$, then $(1, 0)^\top \mapsto (\cos\theta, \sin\theta)^\top$. If $(0, 1)^\top$ is rotated by $\theta$, then $(1, 0)^\top \mapsto (\cos(\theta + \frac{\pi}{2}), \sin(\theta + \frac{\pi}{2}))^\top = (-\sin\theta, \cos\theta)^\top$. Therefore, rotating $(x, y)^\top$, then
$$\begin{pmatrix}
x \\
y
\end{pmatrix} \mapsto \begin{pmatrix}
x\cos\theta - y\sin\theta \\
x\sin\theta + y\cos\theta
\end{pmatrix} = \begin{pmatrix}
\cos\theta & -\sin\theta \\
\sin\theta & \cos\theta
\end{pmatrix}\begin{pmatrix}
x \\
y`
\end{pmatrix}$$
Therefore, the matrix
$$\begin{pmatrix}
\cos\theta & -\sin\theta \\
\sin\theta & \cos\theta
\end{pmatrix}$$`
is the linear transformation rotating the $x, y$ plane by $\theta$.
\item[(b)]
Lemma: Let
$$\mathcal{M} = \begin{pmatrix}
\cos\theta & -\sin\theta \\
\sin\theta & \cos\theta
\end{pmatrix}$$
Then for $k \in\mathbb{Z}$, $$\mathcal{M}^k = \begin{pmatrix}
\cos(k\theta) & -\sin(k\theta) \\
\sin(k\theta) & \cos(k\theta)
\end{pmatrix}$$
Proof: The statement is trivial if $k = 1$ and $k = 0$. Suppose for $k - 1 > 1$, the statement is true. Then,
\tiny
$$\mathcal{M}^k = \begin{pmatrix}
\cos((k-1)\theta) & -\sin((k-1)\theta) \\
\sin((k-1)\theta) & \cos((k-1)\theta)
\end{pmatrix}\begin{pmatrix}
\cos\theta & -\sin\theta \\
\sin\theta & \cos\theta
\end{pmatrix}$$
$$= \begin{pmatrix}
\cos\theta\cos((k-1)\theta) - \sin\theta\sin((k - 1)\theta) & -(\sin\theta\cos((k-1)\theta) + \cos\theta\sin((k-1)\theta)) \\
\cos\theta\sin((k-1)\theta) + \sin\theta\cos((k-1)\theta) & \cos\theta\cos((k-1)\theta) - \sin\theta\sin((k-1)\theta)
\end{pmatrix}$$
$$= \begin{pmatrix}
\cos((k-1)\theta + \theta) & -\sin((k-1)\theta + \theta) \\
\sin((k-1)\theta + \theta) & \cos((k-1)\theta + \theta)
\end{pmatrix} = \begin{pmatrix}
\cos(k\theta) & -\sin(k\theta) \\
\sin(k\theta) & \cos(k\theta)
\end{pmatrix}$$
\normalsize
And, for $k < 0$,
$$\begin{pmatrix}
\cos(k\theta) & -\sin(k\theta) \\
\sin(k\theta) & \cos(k\theta)
\end{pmatrix}\begin{pmatrix}
\cos(-k\theta) & -\sin(-k\theta) \\
\sin(-k\theta) & \cos(-k\theta) 
\end{pmatrix}$$
$$= \begin{pmatrix}
\cos^2(k\theta) + \sin^2(k\theta) & \cos(k\theta)\sin(k\theta) - \sin(k\theta)\cos(k\theta) \\
\sin(k\theta)\cos(k\theta) - \sin(k\theta)\cos(k\theta) & \sin^2(k\theta) + cos^2(k\theta)
\end{pmatrix}$$
$$ = \begin{pmatrix}
1 & 0 \\
0 & 1
\end{pmatrix}$$
Thus, the statement is true for $k < 0$.

Note that 
$$\varphi(r)^n = \begin{pmatrix}
\cos\theta & -\sin\theta \\
\sin\theta & \cos\theta
\end{pmatrix}^n = \begin{pmatrix}
\cos(n\theta) & -\sin(n\theta) \\
\sin(n\theta) & \cos(n\theta)
\end{pmatrix} = \begin{pmatrix}
1 & 0 \\
0 & 1
\end{pmatrix} = \varphi(1)$$
$$\varphi(s)^2 = \begin{pmatrix}
0 & 1 \\
1 & 0
\end{pmatrix}^2 = \begin{pmatrix}
1 & 0 \\
0 & 1
\end{pmatrix} = \varphi(1)$$
$$\varphi(r)\varphi(s) = \begin{pmatrix}
\cos\theta & -\sin\theta \\
\sin\theta & \cos\theta
\end{pmatrix}\begin{pmatrix}
0 & 1 \\
1 & 0
\end{pmatrix} = \begin{pmatrix}
-\sin\theta & \cos\theta \\
\cos\theta & \sin\theta
\end{pmatrix}$$
$$= \begin{pmatrix}
0 & 1 \\
1 & 0
\end{pmatrix}\begin{pmatrix}
\cos(-\theta) & -\sin(-\theta) \\
\sin(-\theta) & \cos(-\theta)
\end{pmatrix} = \varphi(s)\varphi(r)^{-1}$$
All relations of $D_{2n}$ are satisfied by $\varphi$. Then we can extend $\varphi$ to be:
$$\varphi(s^ar^b) = \begin{pmatrix}
0 & 1 \\
1 & 0
\end{pmatrix}^a\begin{pmatrix}
\cos\theta & -\sin\theta \\
\sin\theta & \cos\theta
\end{pmatrix}^b$$
For $0 \leq a < 2$, $0 \leq b < n$. And
$$\varphi(s^ar^bs^cr^d) = \varphi(s^{a+c}r^{d-b}) = \begin{pmatrix}
0 & 1 \\
1 & 0
\end{pmatrix}^{a+c}\begin{pmatrix}
\cos\theta & -\sin\theta \\
\sin\theta & \cos\theta
\end{pmatrix}^{d-b}$$
$$= \begin{pmatrix}
0 & 1 \\
1 & 0
\end{pmatrix}^a\begin{pmatrix}
\cos\theta & -\sin\theta \\
\sin\theta & \cos\theta
\end{pmatrix}^b\begin{pmatrix}
0 & 1 \\
1 & 0
\end{pmatrix}^c\begin{pmatrix}
\cos\theta & -\sin\theta \\
\sin\theta & \cos\theta
\end{pmatrix}^d = \varphi(s^ar^b)\varphi(s^cr^d)$$
Therefore, $\varphi$ is a homomorphism of $D_{2n}$ onto $GL_2(\mathbb{R})$.
\item[(c)]
Let $a = r^i$, for $0 \leq i < n$. Then
$$\varphi(a) = \begin{pmatrix}
\cos(i\theta) & -\sin(i\theta) \\
\sin(i\theta) & \cos(i\theta)
\end{pmatrix}$$
Note that $\varphi(a) \neq I$ unless $i = 0$. Therefore, for each distinct $r^i$, $\varphi(r^i)$ is also distinct. Let $b = sr^i$, for $0 \leq i < n$. Then
$$\varphi(b) = \begin{pmatrix}
0 & 1 \\
1 & 0
\end{pmatrix}\begin{pmatrix}
\cos(i\theta) & -\sin(i\theta) \\
\sin(i\theta) & \cos(i\theta)
\end{pmatrix} = \begin{pmatrix}
\sin(i\theta) & \cos(i\theta) \\
\cos(i\theta) & -\sin(i\theta)
\end{pmatrix}$$
Note that for each distinct $b$, $\varphi(b)$ is also distinct. Furthermore, for any $a$, suppose $\varphi(a) = \varphi(s)$. Then for some $i, k, m$,
$$\cos(i\theta) = \sin\theta = \cos(\frac{\pi}{2} - \theta) \rightarrow (i + 1)\theta = \frac{\pi}{2} + 2\pi k \rightarrow i = \frac{n}{4} + nk - 1$$
And
$$\cos(i\theta) = -\sin(\theta) = \cos(\frac{\pi}{2} + \theta) \rightarrow (i - 1)\theta = \frac{\pi}{2} +2\pi m \rightarrow i = \frac{n}{4} + nm + 1$$
Since $0 < i < n$, then $k = m = 0$, and therefore we have a contradiction, so $\varphi(a) \neq \varphi(s)$, and also for any $b$, $\varphi(a) \neq \varphi(b)$.
Therefore, $\varphi$ is injective.
\end{itemize}
\item[(26)]
Note that
$$\varphi(i)^2 = \begin{pmatrix}
\sqrt{-1} & 0 \\
0 & -\sqrt{-1}
\end{pmatrix}^2 = -I$$
$$\varphi(j)^2 = \begin{pmatrix}
0 & -1 \\
1 & 0
\end{pmatrix}^2 = -I$$
$$\varphi(i)^4 = \varphi(j)^4 = I$$
$$\varphi(i)\varphi(j) = \begin{pmatrix}
0 & -\sqrt{-1} \\
-\sqrt{-1} & 0
\end{pmatrix}$$
$$\varphi(j)\varphi(i) = \begin{pmatrix}
0 & \sqrt{-1} \\
\sqrt{-1} & 0
\end{pmatrix} = (\varphi(i)\varphi(j))^{-1}$$
Thus, the relations of $Q_8$ are preserved, and $\varphi$ extends to a homomorphism. In particular, we can extend $\varphi$ to be:
$$\varphi(1) = I$$
$$\varphi(-1) = -I$$
$$\varphi(-i) = \begin{pmatrix}
-\sqrt{-1} & 0 \\
0 & \sqrt{-1}
\end{pmatrix}$$
$$\varphi(-j) = \begin{pmatrix}
0 & 1 \\
-1 & 0
\end{pmatrix}$$
$$\varphi(k) = \begin{pmatrix}
0 & -\sqrt{-1} \\
-\sqrt{-1} & 0
\end{pmatrix}$$
$$\varphi(-k) = \begin{pmatrix}
0 & \sqrt{-1} \\
\sqrt{-1} & 0
\end{pmatrix}$$
By construction, $\varphi$ is a homomorphism, and is injective.
\end{itemize}
\end{document}