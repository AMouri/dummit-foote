\documentclass[12pt]{article}
\usepackage{amssymb}
\begin{document}
\title{Introduction to Groups - Homomorphisms and Isomorphisms}
\author{Alec Mouri}

\maketitle
\section*{Exercises}
\begin{itemize}
\item[(1)]
\begin{itemize}
\item[(a)]
Suppose $n = 1$. Clearly, $\varphi(x) = \varphi(x)$. Suppose for some $n \in \mathbb{Z}^+$ that $\varphi(x^n) = \varphi(x)^n$. Then:
$$\varphi(x^{n+1}) = \varphi(x^nx) = \varphi(x^n)\varphi(x) = \varphi(x)^n\varphi(x) = \varphi(x)^{n+1}$$
\item[(b)]
Lemma: $\varphi(1) = 1$. Proof: for some $x \in G$:
$$\varphi(x) = \varphi(x)\varphi(1) \rightarrow 1 = \varphi(1)$$
Therefore, by the Lemma,
$$1 = \varphi(1) = \varphi(xx^{-1}) = \varphi(x)\varphi(x^{-1}) \rightarrow \varphi(x^{-1}) = \varphi(x)^{-1}$$
Let $n = 0$. Then $\varphi(x^0) = \varphi(1) = 1 = \varphi(x)^0$

Let $n < 0$. Then
$$\varphi(x^n) = \varphi((x^{-n})^{-1}) = \varphi(x^{-n})^{-1} = (\varphi(x)^{-n})^{-1} = \varphi(x)^n$$
\end{itemize}
\item[(2)]
Suppose $\varphi$ is an isomorphism. Suppose $|\varphi(x)| = \infty$, and for sake of contradiction that $|x| = n < \infty$. Then $\varphi(x)^n = \varphi(x^n) = \varphi(1) = 1 \rightarrow |\varphi(x)| \leq n$, a contradiction. Thus, $|x| = \infty$. Suppose $|x| = \infty$, and for sake of contradiction that $|\varphi(x)| = n < \infty$. Then $\varphi(1) = 1 = \varphi(x)^n = \varphi(x^n) \rightarrow 1 = x^n \rightarrow |x| \leq n$, a contradiction. Thus $|\varphi(x)| = \infty$. Suppose $|x|$ and $|\varphi(x)|$ are finite. Then we have
$$1 = \varphi(1) = \varphi(x^{|x|}) = \varphi(x)^{|x|}$$
Suppose for some $1 \leq i < |x|$, that $\varphi(x)^i = 1$. Then 
$$1 = \varphi(x)^i = \varphi(x^i) \rightarrow x^i = 1$$
But, this implies that $|x|$ is not the order of $x$, a contradiction. Therefore, no such $i$ exists such that $\varphi(x)^i = 1$, and therefore $|\varphi(x)| = |x|$.

Let $G_n$ be the subset of $G$ containing all elements of $G$ that have order $n$, and let $H_n$ be the subset of $H$ containing all elements of $H$ that have order $n$. Since $\varphi$ is a bijection, and for all $x \in G_n$, $\varphi(x) \in H_n$, then $|G_n| = |H_n|$.

The above result does not hold if $\varphi$ is a homomorphism, but not an isomorphism. Consider the Quaternion group $Q_8$. Consider the cyclic subgroup $q = \left\lbrace i^n | n \in \mathbb{Z} \right\rbrace = \left\lbrace 1, -1, i, -i \right\rbrace$ generated by $i$. Define $\varphi: q \rightarrow Q_8$ such that for all $x \in q$: $\varphi(x) = x$. Clearly, $\varphi$ is a homomorphism, and not an isomorphism. $q$ contains 2 elements of order 4, $i$ and $-i$. But, $Q_8$ contains 6 elements of order 4: $i, -i, j, -j, k, -k$.
\item[(3)]
Suppose $G$ is abelian. Since $\varphi$ is surjective, then for $h_1, h_2 \in H$, there exists some $g_1, g_2 \in G$, $\varphi(g_1) = h_1$ and $\varphi(g_2) = h_2$. Then
$$h_1h_2 = \varphi(g_1)\varphi(g_2) = \varphi(g_1g_2) = \varphi(g_2g_1) = \varphi(g_2)\varphi(g_1) = h_2h_1$$
Thus $H$ is abelian. Now suppose $H$ is abelian. Since $\varphi$ is injective, then for some $g_1, g_2 \in G$ where $g_1 \neq g_2$, then $\varphi(g_1) \neq \varphi(g_2)$. Therefore,
$$\varphi(g_1g_2) = \varphi(g_1)\varphi(g_2) = \varphi(g_2)\varphi(g_1) = \varphi(g_2g_1)$$
Since $\varphi$ is injective, then $g_1g_2 = g_2g_1$, and $G$ is abelian.

If $\varphi$ is a homomorphism, then $\varphi$ must be surjective for $H$ to be abelian if $G$ is abelian. See the cyclic subgroup of the quaternion group in the solution for exercise 2: $\varphi$ from $q$ to $Q_8$ is not surjective, and $q$ is abelian, but $Q_8$ is not abelian.
\item[(4)]
Suppose $\mathbb{R} - \left\lbrace 0 \right\rbrace$ and $\mathbb{C} - \left\lbrace 0 \right\rbrace$ are isomorphic. From Exercise 2, for each $n \geq 1$, two isomorphic groups have the same number of elements with order $n$. Note that for $x \in \mathbb{R} - \left\lbrace 0 \right\rbrace$, if $x = 1$, then $|x| = 1$, and if $x = -1$, then $|x| = 2$. If $|x| > 1$ (where the bars denote the magnitude of $x$), then $|x|^m > 1$ for all $m \geq 1$, so the order of $x$ is infinite. Similarly, if $|x| < 1$, then $|x|^m < 1$ for all $m \geq $, so the order of $x$ is again infinite. However, consider $i \in \mathbb{C} - \left\lbrace 0 \right\rbrace$. $i$ has order 4, but $\mathbb{R} - \left\lbrace 0 \right\rbrace$ has no elements of order 4. Therefore, the groups are not isomorphic.
\item[(5)]
Since $\mathbb{Q}$ is countable and $\mathbb{R}$ is not, then there does not exist a bijection between $\mathbb{Q}$ and $\mathbb{R}$, so therefore they are not isomorphic.
\item[(6)]
Lemma: Suppose $\varphi: G \rightarrow H$ is a surjective homomorphism. If $G$ is generated by one element, then $H$ must also be generated by one element. Proof: Let $g$ be the generator of $G$. For any $a \in G$, then $a = g^k$ for some $k \in \mathbb{Z}$. For any $b \in H$, then for some $c \in G$, $\varphi(c) = b$. Therefore,
$$b = \varphi(c) = \varphi(g^k) = \varphi(g)^k$$
That is, $\varphi(g)$ generates $H$.

Suppose $\varphi$ is a bijection between $\mathbb{Z}$ and $\mathbb{Q}$, and that they are isomorphic. Note that 1 generates $\mathbb{Z}$, so therefore by the Lemma some $x \in \mathbb{Q}$ must be a generator. Suppose $\frac{i}{j}$ generates $\mathbb{Q}$. Note that since $\frac{1}{j} \in \mathbb{Q}$, and for all $k \in \mathbb{Z}$, $\frac{ki}{j} \neq \frac{1}{j}$ unless $i = 1$. Consider $\frac{1}{a} \in \mathbb{Q}$, where $a > j$. But for any $k \in \mathbb{Z}, \frac{k}{j} \neq \frac{1}{a}$. Therefore, $\frac{1}{j}$ cannot generate $\mathbb{Q}$. Then $\mathbb{Z}$ and $\mathbb{Q}$ cannot be isomorphic.
\item[(7)]
There are 2 elements of $D_8$ that have order 4: $r$ and $r^3$. But there are 6 elements of $Q_8$ that have order 4: $i, -i, j, -j, k$, and $-k$. Therefore by Exercise 2, $D_8$ and $Q_8$ are not isomorphic.
\item[(8)]
Note that $|S_n| = n!$, and $|S_m| = m!$. Thus if $n \neq m$, then $|S_n| \neq |S_m|$, so therefore $S_n$ and $S_m$ are not isomorphic.
\item[(9)]
There is at least 1 element of $D_{24}$ that has order 12: $r$. But, there are no elements of $S_4$ that have order 12. In particular, the maximum order of any element of $S_4$ is 4. Thus, $D_{24}$ and $S_4$ are not isomorphic.
\item[(10)] 
\end{itemize}
\end{document}