\documentclass[12pt]{article}
\usepackage{amsmath, amssymb}
\begin{document}
\title{Introduction to Groups - Dihedral Groups}
\author{Alec Mouri}

\maketitle
\section*{D2n Calculations}
For these calculations, we shall re-index each vertex so that the vertices are ordered from 0 to $n - 1$, as opposed to ordering them from 1 to $n$.
\begin{itemize}
\item[(1)] 
If $r$ is a clockwise rotation about the origin through $\frac{2\pi}{n}$, then the vertex $i$ is sent to vertex $i + 1$ modulo $n$. Furthermore, $r^k$ is the clockwise rotation about the origin through $\frac{2\pi k}{n}$, and so the vertex $i$ is sent to the vertex $i + k$ modulo $n$. Clearly then, $1, r, r^2, ..., r^{n - 1}$ are unique. And $r^n$ is the clockwise rotation through $2\pi$, so the vertex $i$ is sent to the vertex $i + n \equiv i \mod n$. Therefore, $r^n = 1 \rightarrow |r| = n$.
\item[(2)]
If $s$ is a reflection about a line of symmetry, then any vertex $i$ is sent to some vertex $j$, and $j$ is sent to $i$. Therefore, $s^2$ sends vertex $i$ to itself. Since $s$ sends vertex 1 to vertex $n - 1$, then $|s| = 2$.
\item[(3)]
The reflection $s$ sends vertex 0 to itself, and vertex 1 to vertex $n - 1$. For $k \in \mathbb{Z}^+$, $r^k$ sends vertex 0 to vertex $k \mod n$. If $k \not \equiv 0 \mod n$, then vertex 0 is not sent to itself, so then $r^k \neq s$. If $k \equiv 0 \mod n$, then vertex 0 is sent to itself. But vertex 1 is sent to itself, not to $n - 1$. Therefore, $s \neq r^k$ for any $k$.
\item[(4)]
Suppose $sr^i = sr^j$. Then
$$s^2r^i = s^2r^j \rightarrow r^i = r^j$$
But by (1), $r^i \neq r^j$. Therefore $sr^i \neq
sr^j$. It then follows from (1), (2), and (3) that
$$D_{2n} = \left\lbrace 1, r, r^2, ..., r^{n - 1}, s, sr, s^2, ..., sr^{n - 1}\right\rbrace$$
\item[(5)]
For this problem, we shall 0 index the vertices. If $s$ is a reflection about the line of symmetry passing through vertex 0, then vertex $i$ is sent to vertex $n - i \mod n$. So for vertex $i$, the permutation $rs$ first sends $i$ to $n - i$, then sends vertex $n - i$ to vertex $n - i + 1$. The permutation $sr^{-1}$ first sends $i$ to $i - 1$, then sends vertex $i - 1$ to vertex $n - (i - 1) = n - i + 1$. So therefore $rs = sr^{-1}$.
\item[(6)]
If $i = 0$, then
$$r^0s = s = sr^{-0}$$
Suppose for $0 \leq k < i$ that
$$r^ks = sr^{-k}$$
Then 
$$r^is = rr^{i - 1}s = r(r^{i - 1}s) = r(sr^{-(i - 1)})$$
$$= (rs)r^{-(i - 1)} = (sr)r^{-(i - 1)} = sr^{-i}$$
\end{itemize}
\section*{Exercises}
\begin{itemize}
\item[(1)]
\begin{itemize}
\item[(a)]
$$1^1 = 1 \rightarrow |1| = 1$$
$$(r)^3 = 1 \rightarrow |r| = 3$$
$$(r^2)^3 = 1 \rightarrow |r^2| = 3$$
$$s^2 = 1 \rightarrow |s| = 2$$
$$(sr)^2 = srr^{-1}s = s^2 = 1 \rightarrow |sr| = 2$$
$$(sr^2)^2 = sr^2r^{-2}s = s^2 = 1 \rightarrow |sr^2| = 2$$
\item[(b)]
$$1^1 = 1 \rightarrow |1| = 1$$
$$(r)^4 = 1 \rightarrow |r| = 4$$
$$(r^2)^2 = 1 \rightarrow |r^2| = 2$$
$$(r^3)^4 = 1 \rightarrow |r^3| = 4$$
$$s^2 = 1 \rightarrow |s| = 2$$
$$(sr)^2 = srr^{-1}s = s^2 = 1 \rightarrow |sr| = 2$$
$$(sr^2)^2 = sr^2r^{-2}s = s^2 = 1 \rightarrow |sr^2| = 2$$
$$(sr^3)^2 = sr^3r^{-3}s = s^2 = 1 \rightarrow |sr^3| = 2$$
\item[(c)]
$$1^1 = 1 \rightarrow |1| = 1$$
$$(r)^5 = 1 \rightarrow |r| = 5$$
$$(r^2)^5 = 1 \rightarrow |r^2| = 5$$
$$(r^3)^5 = 1 \rightarrow |r^3| = 5$$
$$(r^4)^5 = 1 \rightarrow |r^4| = 5$$
$$s^2 = 1 \rightarrow |s| = 2$$
$$(sr)^2 = srr^{-1}s = s^2 = 1 \rightarrow |sr| = 2$$
$$(sr^2)^2 = sr^2r^{-2}s = s^2 = 1 \rightarrow |sr^2| = 2$$
$$(sr^3)^2 = sr^3r^{-3}s = s^2 = 1 \rightarrow |sr^3| = 2$$
$$(sr^4)^2 = sr^4r^{-4}s = s^2 = 1 \rightarrow |sr^4| = 2$$
\end{itemize}
\item[(2)]
Let $x \in D_{2n}$. Then $x = s^ir^j$ for $i = 0, 1$ and $0 \leq j < n$. $x$ must not be a power of $r$. If $i = 0$, then $x$ is a power, so therefore $i = 1$, and $x = sr^j$. Therefore
$$rx = rsr^j = (rs)r^j = (sr^{-1})r^j = sr^{j - 1} = sr^jr^{-1} = xr^{-1}$$
\item[(3)]
Let $x \in D_{2n}$ such that $x$ is not a power of $r$. Then $x = sr^j$ for $0 \leq j < n$. Clearly, $x \neq 1$. And,
$$x^2 = (sr^j)^2 = sr^jsr^j = s(sr^{-j})r^j = s^2 = 1$$
Thus, $|x| = 2$. Note that $s \in <s, sr>$. Since $r = ssr$, then $r \in <s, sr>$. Therefore $D_{sn} = <s, sr>$.
\end{itemize}
\end{document}