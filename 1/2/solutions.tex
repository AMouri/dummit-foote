\documentclass[12pt]{article}
\usepackage{amsmath, amssymb}
\begin{document}
\title{Introduction to Groups - Dihedral Groups}
\author{Alec Mouri}

\maketitle
\section*{D2n Calculations}
For these calculations, we shall re-index each vertex so that the vertices are ordered from 0 to $n - 1$, as opposed to ordering them from 1 to $n$.
\begin{itemize}
\item[(1)] 
If $r$ is a clockwise rotation about the origin through $\frac{2\pi}{n}$, then the vertex $i$ is sent to vertex $i + 1$ modulo $n$. Furthermore, $r^k$ is the clockwise rotation about the origin through $\frac{2\pi k}{n}$, and so the vertex $i$ is sent to the vertex $i + k$ modulo $n$. Clearly then, $1, r, r^2, ..., r^{n - 1}$ are unique. And $r^n$ is the clockwise rotation through $2\pi$, so the vertex $i$ is sent to the vertex $i + n \equiv i \mod n$. Therefore, $r^n = 1 \rightarrow |r| = n$.
\item[(2)]
If $s$ is a reflection about a line of symmetry, then any vertex $i$ is sent to some vertex $j$, and $j$ is sent to $i$. Therefore, $s^2$ sends vertex $i$ to itself. Since $s$ sends vertex 1 to vertex $n - 1$, then $|s| = 2$.
\item[(3)]
The reflection $s$ sends vertex 0 to itself, and vertex 1 to vertex $n - 1$. For $k \in \mathbb{Z}^+$, $r^k$ sends vertex 0 to vertex $k \mod n$. If $k \not \equiv 0 \mod n$, then vertex 0 is not sent to itself, so then $r^k \neq s$. If $k \equiv 0 \mod n$, then vertex 0 is sent to itself. But vertex 1 is sent to itself, not to $n - 1$. Therefore, $s \neq r^k$ for any $k$.
\item[(4)]
Suppose $sr^i = sr^j$. Then
$$s^2r^i = s^2r^j \rightarrow r^i = r^j$$
But by (1), $r^i \neq r^j$. Therefore $sr^i \neq
sr^j$. It then follows from (1), (2), and (3) that
$$D_{2n} = \left\lbrace 1, r, r^2, ..., r^{n - 1}, s, sr, s^2, ..., sr^{n - 1}\right\rbrace$$
\item[(5)]
For this problem, we shall 0 index the vertices. If $s$ is a reflection about the line of symmetry passing through vertex 0, then vertex $i$ is sent to vertex $n - i \mod n$. So for vertex $i$, the permutation $rs$ first sends $i$ to $n - i$, then sends vertex $n - i$ to vertex $n - i + 1$. The permutation $sr^{-1}$ first sends $i$ to $i - 1$, then sends vertex $i - 1$ to vertex $n - (i - 1) = n - i + 1$. So therefore $rs = sr^{-1}$.
\item[(6)]
If $i = 0$, then
$$r^0s = s = sr^{-0}$$
Suppose for $0 \leq k < i$ that
$$r^ks = sr^{-k}$$
Then 
$$r^is = rr^{i - 1}s = r(r^{i - 1}s) = r(sr^{-(i - 1)})$$
$$= (rs)r^{-(i - 1)} = (sr)r^{-(i - 1)} = sr^{-i}$$
\end{itemize}
\section*{Exercises}
\begin{itemize}
\item[(1)]
\begin{itemize}
\item[(a)]
$$1^1 = 1 \rightarrow |1| = 1$$
$$(r)^3 = 1 \rightarrow |r| = 3$$
$$(r^2)^3 = 1 \rightarrow |r^2| = 3$$
$$s^2 = 1 \rightarrow |s| = 2$$
$$(sr)^2 = srr^{-1}s = s^2 = 1 \rightarrow |sr| = 2$$
$$(sr^2)^2 = sr^2r^{-2}s = s^2 = 1 \rightarrow |sr^2| = 2$$
\item[(b)]
$$1^1 = 1 \rightarrow |1| = 1$$
$$(r)^4 = 1 \rightarrow |r| = 4$$
$$(r^2)^2 = 1 \rightarrow |r^2| = 2$$
$$(r^3)^4 = 1 \rightarrow |r^3| = 4$$
$$s^2 = 1 \rightarrow |s| = 2$$
$$(sr)^2 = srr^{-1}s = s^2 = 1 \rightarrow |sr| = 2$$
$$(sr^2)^2 = sr^2r^{-2}s = s^2 = 1 \rightarrow |sr^2| = 2$$
$$(sr^3)^2 = sr^3r^{-3}s = s^2 = 1 \rightarrow |sr^3| = 2$$
\item[(c)]
$$1^1 = 1 \rightarrow |1| = 1$$
$$(r)^5 = 1 \rightarrow |r| = 5$$
$$(r^2)^5 = 1 \rightarrow |r^2| = 5$$
$$(r^3)^5 = 1 \rightarrow |r^3| = 5$$
$$(r^4)^5 = 1 \rightarrow |r^4| = 5$$
$$s^2 = 1 \rightarrow |s| = 2$$
$$(sr)^2 = srr^{-1}s = s^2 = 1 \rightarrow |sr| = 2$$
$$(sr^2)^2 = sr^2r^{-2}s = s^2 = 1 \rightarrow |sr^2| = 2$$
$$(sr^3)^2 = sr^3r^{-3}s = s^2 = 1 \rightarrow |sr^3| = 2$$
$$(sr^4)^2 = sr^4r^{-4}s = s^2 = 1 \rightarrow |sr^4| = 2$$
\end{itemize}
\item[(2)]
Let $x \in D_{2n}$. Then $x = s^ir^j$ for $i = 0, 1$ and $0 \leq j < n$. $x$ must not be a power of $r$. If $i = 0$, then $x$ is a power, so therefore $i = 1$, and $x = sr^j$. Therefore
$$rx = rsr^j = (rs)r^j = (sr^{-1})r^j = sr^{j - 1} = sr^jr^{-1} = xr^{-1}$$
\item[(3)]
Let $x \in D_{2n}$ such that $x$ is not a power of $r$. Then $x = sr^j$ for $0 \leq j < n$. Clearly, $x \neq 1$. And,
$$x^2 = (sr^j)^2 = sr^jsr^j = s(sr^{-j})r^j = s^2 = 1$$
Thus, $|x| = 2$. Note that $s \in <s, sr>$. Since $r = ssr$, then $r \in <s, sr>$. Therefore $D_{sn} = <s, sr>$.
\item[(4)]
Let $n = 2k$ and $z = r^k$. Clearly $z \neq 1$. And
$$z^2 = (r^k)^2 = r^{2k} = r^n = 1$$
Therefore, $z$ has order 2, and in particular,
$$r^k = r^{-k}$$.
Let $a = s^ir^j \in D_{2n}$. where $s = 0, 1$ and $0 \leq j \leq n - 1$. Then
$$za = r^ks^ir^j = s^ir^{-k}r^j = s^ir^jr^{-k} = az$$
So $z$ commutes with $a$. Consider $x = sr^i \in D_{2n}$. Then
$$xr = sr^ir = r^{-1}sr^{i-1}r = r^{-1}sr^i$$
And
$$r^{-1}sr^i = sr^{i - 1}$$
But then
$$i + 1 \equiv i - 1 \mod n$$
Which is a contradiction, so $x$ cannot commute. Now consider $y = r^i \in D_{2n}$. Then
$$ysr = r^isr = sr^{-i}r = sry^{-1}$$
Suppose for sake of contradiction that $y$ commutes with $sr$. Then
$$sry = ysr = sry^{-1} \rightarrow y = y^{-1} \rightarrow r^i = r^{-i}$$
So either $i = 0$ or $i = k$. So only the identity and $r^k$ commute.
\item[(5)]
Let $n$ be odd. Let $x = sr^i \in D_{2n}$. Then
$$xr = sr^ir = r^{-1}sr^{i - 1}r = r^{-1}sr^i$$
And
$$r^{-i}sr^i = sr^{i - 1}$$
But then
$$i + 1 \equiv i - 1 \mod n$$
Which is a contradiction, so $x$ cannot commute. Now consider $y = r^i \in D_{2n}$. Then
$$ysr = r^isr = sr^{-i}r = sry^{-1}$$
Suppose contradiction that $y$ commutes with $sr$. Then
$$sry = ysr = sry^{-1} \rightarrow y = y^{-1} \rightarrow r^i = r^{-i}$$
So then $2i \equiv 0 \mod n$. But since $n$ is odd, then $i = 0$. Thus, only the identity can commute with elements of $D_{2n}$.
\item[(6)]
Let $t = xy$. Since $|x| = |y| = 2$, then
$$xyyx = xx = 1 \rightarrow t^{-1} = yx$$
Then
$$tx = xyx = xt^{-1}$$
\item[(7)]
Let $a = s$ and $b = sr$. If $a^2 = b^2 = (ab)^n = 1$, then $s^2 = a^2 = 1$, and $1 = (ab)^n = (ssr)^n = r^n$. And, $b = b^{-1} \rightarrow sr = (sr)^{-1} = r^{-1}s^{-1} = r^{-1}s$. \\
If $s^2 = r^n = 1$ and $sr = r^{-1}s$, then $a^2 = s^2 = 1$, and $(ab)^n = (ssr)^n = r^n = 1$. And, $sr = r^{-1}s \rightarrow (sr)^2 = 1 \rightarrow b^2 = 1$.
\item[(8)]
The cyclic subgroup generated by $r$ is $\mathcal{A} = \left\lbrace r^n | n \in \mathbb{Z} \right\rbrace$. Since the elements of $\mathcal{A}$ are $1, r, ..., r^{n-1}$, $|\mathcal{A}| = n$.
\item[(9)]
Choose an orientation of a tetrahedron. For a particular edge, label its vertices $1, 2$. Consider a novel permutation $\sigma$ of vertices. There are 4 possibilities for $\sigma(1)$. If $\sigma(1)$ is fixed, then there are 3 possibilities for $\sigma(2)$. Therefore, $|G| = 4 \cdot 3 = 12$. 
\item[(10)]
Choose an orientation of a cube. For a particular edge, label its vertices $1, 2$. Consider a novel permutation $\sigma$ of vertices. There are 8 possibilities for $\sigma(1)$. If $\sigma(1)$ is fixed, then there are 3 possibilities for $\sigma(2)$. Therefore, $|G| = 8 \cdot 3 = 24$. 
\item[(11)]
Choose an orientation of an octahedron. For a particular edge, label its vertices $1, 2$. Consider a novel permutation $\sigma$ of vertices. There are 6 possibilities for $\sigma(1)$. If $\sigma(1)$ is fixed, then there are 4 possibilities for $\sigma(2)$. Therefore, $|G| = 6 \cdot 4 = 24$. 
\item[(12)]
Choose an orientation of a dodecahedron. For a particular edge, label its vertices $1, 2$. Consider a novel permutation $\sigma$ of vertices. There are 20 possibilities for $\sigma(1)$. If $\sigma(1)$ is fixed, then there are 3 possibilities for $\sigma(2)$. Therefore, $|G| = 20 \cdot 3 = 60$. 
\item[(13)]
Choose an orientation of an icosahedron. For a particular edge, label its vertices $1, 2$. Consider a novel permutation $\sigma$ of vertices. There are 12 possibilities for $\sigma(1)$. If $\sigma(1)$ is fixed, then there are 5 possibilities for $\sigma(2)$.Therefore, $|G| = 12 \cdot 5 = 60$.
\item[(14)]
Since all integers are finite sums of 1s and -1s, then $\mathbb{Z} = <1>$.
\item[(15)]
All integers modulo $n$ are finite sums of 1s, with the relation $n = 0$. Therefore, $\mathbb{Z}/n\mathbb{Z} = <1 | 1^n = 0>$
\item[(16)]
Let $x_1^2 = y_1^2 = (x_1y_1)^2 = 1$, $r = x_1$, and $s = y_1$. Then $s^2 = y_1^2 = 1$, and $r^2 = x_1^2 = 1$. And 
$$rs = x_1y_1 = (x_1y_1)^{-1} = y_1^{-1}x_1^{-1} = s^{-1}r^{-1} = sr^{-1}$$
And
$$rs = sr^{-1} \rightarrow rsrs^{-1} = 1 \rightarrow (rs)^2 = 1 \rightarrow (x_1y_1) = 1$$
\item[(17)]
$$X_{2n} = < x, y | x^n = y^2 = 1, xy = yx^2 >$$
\begin{itemize}
\item[(a)]
Let $n = 3k$. Then
$$x = xy^2 = (xy)y = yx^2y = (yx)(xy) = (yx)yx^2 = y(xy)x^2$$
$$= y(yx^2)x^2 = y^2x^4 = x^4$$
Thus, $x^3 = 1$. Therefore, the elements of $X_{6k}$ are
$$1, x, x^2, y, yx, yx^2$$
Let $x = r$ and $y = s$. Then $x^n = r^n = 1$, $y^2 = s^2 = 1$, and $xy = rs = sr^{-1} = sr^2 = yx^2$. Also, $r^n = x^n = 1$, $s^2 = y^2 = 1$, and $rs = xy = yx^2 = sr^2 = sr^{-1}$. Thus, $X_{2n} = D_{6}$
\item[(b)]
Suppose $(n, 3) = 1$. If $x \neq 1$, then $x^n = x^3 \rightarrow n \equiv 0 \mod 3$, ie. $n = 3k$ for some $k \in \mathbb{Z}^+$. But this contradicts $(n, 3) = 1$, so therefore $x = 1$. Then clearly $X_{2n}$ is order 2, and has two elements:, $1, y$.
\end{itemize}
\item[(18)]
$$Y = <u, v | u^4 = v^3 = 1, uv = v^2u^2>$$
\begin{itemize}
\item[(a)]
$$v^3 = 1 \rightarrow v^2 = v^{-1}$$
\item[(b)]
$$v^2u^3v = v^2u^2v^2u^2 = uvv^2u^2 = uv^3u^2 = u^7 = u^3$$
Thus
$$v^2u^3v = u^3 \rightarrow v^{-1}u^3v = u^3 \rightarrow u^3v = vu^3$$
\item[(c)]
$$u^4 = 1 \rightarrow u^8 = 1 \rightarrow u^9 = u$$
Then
$$vu = vu^9 = u^3vu^6 = u^6vu^3 = u^9v = uv$$
\item[(d)]
Note that for any $k \geq 2$, $v^ku^k = v^{k-1}uvu^{k-1} = v^{k+1}u^{k+1}$. Then
$$uv = v^2u^2 = u^{12}v^{12} = 1$$
\item[(e)]
$$u^4v^3 = 1 \rightarrow u^3v^2 = 1 \rightarrow u^2v = 1 \rightarrow u = 1$$
Thus,
$$1 = uv = 1v = v$$
Since $u = 1$ and $v = 1$, then the elements of $Y$ can only be generated by the identity. That is, for all $i, j \geq 0$, $u^iv^j = 1$. Therefore, $Y = 1$.
\end{itemize}
\end{itemize}
\end{document}